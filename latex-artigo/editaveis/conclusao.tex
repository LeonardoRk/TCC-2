\chapter[Conclusão]{Conclusão}

Como poucos alunos jogaram e apresentam-se os conceitos esperados nos mapas conceituais acredita-se que o resultado da pesquisa é inconclusivo, pois não se sabe se o jogo que ajudou a aprender ou foram apenas as aulas e os estudos por conta própria. A ideia inicial era avaliar a nota da questão da prova geral a qual os alunos realizarão para comparar a turma que jogou com turmas que não jogaram para avaliar se o jogo surtiu efeito ou não no aprendizado.


A respeito do processo de desenvolvimento de software poderiam ter sido feitos mais testes e os realizar junto do desenvolvimento das funcionalidades, onde os casos de teste encontram-se mais frescos na mente. Poderia ter tido um planejamento melhor com o levantamento das funcionalidades do jogo e a descrição dos requisitos. O código pode estar mais limpo e modularizado (refatorado) e o jogo poderia ter sido pensado melhor com mais detalhes em fases e animações.

Para trabalhos futuros podem ser pensadas em mais funcionalidades para o jogo, como a criação de um personagem e ganho de itens para utilizações no jogo.
