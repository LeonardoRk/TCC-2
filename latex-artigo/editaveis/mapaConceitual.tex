\section[Mapas Conceituais]{Mapas conceituais}

Primeiro de tudo, o que é um mapa conceitual (MC) e como fazer um? De acordo com \apud{novak}{vantagensDesvantagensMC}, "mapa conceitual é uma estrutura hierárquica, iniciados por conceitos mais abrangentes, os quais progressivamente vão sendo relacionados com conceitos mais específicos e esclarecendo suas relações de subordinação". 

Um Mapa Conceitual é formado por conceitos e palavras de enlace. Um conceito pode ser uma palavra ou uma expressão chave identificado geralmente em um retângulo ou alguma outra forma e palavras de enlace podem ser uma palavra ou uma expressão que conecta conceitos de modo que dois conceitos conectados por uma palavra ou expessão de enlace é chamado de proposição. Podem ser formadas proposições verdadeiras ou falsas.

O MC pode ser alterado tanto em profundidade quanto em extensão. Profundidade se refere a especificação de algum conceito geral granularizando-o em novas proposições, ou seja, novos conceitos e palavras de enlace. Extensão se refere à adição de novos conceitos, porém não a granularização de um conceito anterior.


Um estudo lido foi o "Mapa conceitual: seu potencial como instrumento avaliativo", ele ajudou a entender vantages e desvantagens de um mapa conceitual. Este foi realizado com 32 alunas de pedagogia do 3º semestre que utilizaram o mapa conceitual para sintetizar informações de muitos textos que foram lidos. Todos os mapas eram apresentados para a turma e havia o debate dos mapas em relação a estar abordando a maioria dos conceitos chaves dos textos ou não. 

As alunas eram identificadas como Ax (sendo x o número de 1 a 32, que é a quantidade de participantes). Foram realizados entrevistas, questionários e também solicitado aos grupos de 3 ou 4 alunas que registrassem 3 vantagens e 3 desvantagens percebidos para avaliar a experiência e o potencial do MC. As vantagens elencadas do mapa foram: 

\begin{itemize}
\item Ajuda identificar as dificuldades de aprendizagem
\end{itemize}

A2 diz que "mapas conceituais tornam os conhecimentos mais claros no que se sabe ou não, porque evidencia o que foi aprendido, mostrando também dúvidas, dificuldades e erros" \cite{vantagensDesvantagensMC}.


\begin{itemize}
\item favorecer a reelaboração de conceitos a sua consequente sedimentação
\end{itemize}

O MC deixa claro a reorganização cognitiva, pois os conceitos conforme são aprofundados e entendidos melhor vão se estendendo e as proposições formadas são alteradas.\cite{vantagensDesvantagensMC}

\begin{itemize}
\item proporcionar feedback quase imediato
\end{itemize}

Pois ao iniciar a elaboração do mapa já percebe-se em quais conceitos há a dificuldade de falar a respeito e de conectar.


\begin{itemize}
\item integração e ampliação dos conhecimentos
\end{itemize}
"O trabalho com mapas conceituais nos levou a aprender a identificar os elementos essenciais e inter-relacioná-los" A8 \cite{vantagensDesvantagensMC}.

Em \cite{vantagensDesvantagensMC} é dito que o mapa conceitual além de ser uma ferramenta avaliativa, também se configurou como estratégia de aprendizagem, vantagem enunciada por 31\% das duplas participantes.
38\% das alunas também declararam que o mapa conceitual possibilita efetivar sucessivas síncreses, análises e sínteses, porém precisa ser discutido em conjunto para sempre aumentar a compreensão.

"Apesar de a aprendizagem implicar a elaboração e a reelaboração do conhecimento pelo educando, ela também permanece refém de interações com os pares e com o professor" \cite[p. 180]{vantagensDesvantagensMC}, ou seja, o aluno não aprende tudo sozinho, são também nas interações com outros envolvidos que as experiências ficam armazenadas e possibilitam chances de aprendizado.

Acima foi citado algumas das vantagens percebidas pelos MCs e também a observação que além da realização do mapa são necessárias interações entre os participantes para a colaboração no aumento do conhecimento e correções de potenciais erros existentes nos mapas de alguns. Abaixo encontra-se observações e obstáculos encontrados no mesmo estudo a respeito do mapa conceitual: 

\begin{itemize}
 \item  \cite{dificuldadesMapaConceitual} cita a dificuldade de corrigir mapas conceituais comparando a questões de múltipla escolha. 
\end{itemize}

 Primeiro porque o mapa pode ser muito extenso. Segundo que demanda tempo para olhar todas as proposições e avaliá-las quanto à sua corretude e a veracidade. Terceiro porque cada mapa é diferente, único e demanda tempo para ser analisado, principalmente se tiver muitos mapas da turma. Devido a estas dificuldades e também pelo fato dos alunos aprenderem mais estando incluídos na correção, pois podem comparar com seus mapas com outros, que os autores \cite{dificuldadesMapaConceitual} disseram ser necessário incluir os alunos na etapa de correção. 


\begin{itemize}
\item Leva tempo e exige certo treino, prática e conhecimento para conseguir organizar as informações existentes em seu cérebro de modo a expô-los em conceito e conectores.
\end{itemize}

Então no ínicio pode ser trabalhoso e complicado fazer um mapa claro e entendível.

Passado o ponto de explicar mapas conceituais, é necessário entender o que ele é e para que serve.
O MC é uma ferramenta de avaliação de conhecimento, que pode ser usado para ver onde as pessoas estão errando nos conceitos e nas proposições e também avaliar o quão extenso é a rede de conhecimentos da pessoa em determinado tema \cite{vantagensDesvantagensMC}.

Este estudo \cite{leiDeNewtonMC} por concordar com \cite{novak} que os MCs contribuem para o ensino-aprendizagem utilizou o mapa conceitual para conhecer o que os alunos sabem sobre Gravitação.


Tendo visto a utilização de mapas conceituais como uma ferramenta de avaliação, deseja-se utilizar o Mapa Conceitual (MC) também neste trabalho como ferramenta que avalia a fixação de conteúdo dos alunos. 

