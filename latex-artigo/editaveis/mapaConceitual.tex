
\section[Teoria Ausubel e Mapa Conceitual]{Teoria Ausubel e Mapa Conceitual}


Na teoria de Ausubel de aprendizagem significativa uma varíavel bastante crucial é a estrutura cognitiva do aprendiz \cite[p. 26]{ausebel}, pois é nela onde o novo conhecimento irá se fixar. O novo conhecimento deve se relacionar de uma maneira não arbitrária e substantiva na estrutura cognitiva \cite[p. 26]{ausebel}.

Acredito que para tornar mais eficiente a absorção do conhecimento dos alunos, para analisar o ensino/aprendizado devemos olhar da esfera do alunos e não dos professores. Como assim? Ao invés do professor ensinar para o aluno, visão essa que torna o aluno como reativo, devemos olhar do ponto em que o aluno aprende com o professor. Desta segunda maneira é uma visão pró-ativa, onde o aluno não tem apenas o professor como única fonte de absorção de conhecimento, mas como uma das fontes. Nesta segunda visão o estudante deve ir atrás de como adquirir o conhecimento e seguir conselhos de pessoas mais experientes (como os professores).

Para mim o mapa conceitual é uma maneira de se alcançar a aprendizagem significativa, mas primeiro de tudo, o que é mapa conceitual (MC)? De acordo com \apud{novak}{vantagensDesvantagensMC}, "mapa conceitual é uma estrutura hierárquica, iniciados por conceitos mais abrangentes, os quais progressivamente vão sendo relacionados com conceitos mais específicos e esclarecendo suas relações de subordinação".

Um Mapa Conceitual é formado por conceitos e palavras de enlace. Um conceito pode ser uma palavra ou uma expressão chave identificado geralmente em um retângulo ou alguma outra forma e palavras de enlace podem ser uma palavra ou uma expressão que conecta conceitos de modo que dois conceitos conectados por uma palavra ou expessão de enlace é chamado de proposição. Podem ser formadas proposições verdadeiras ou falsas, por isso deve-se analisar um mapa conceitual para checar se as proposições formadas não contém erros.

O MC pode ser alterado tanto em profundidade quanto em extensão. Profundidade se refere a especificação de algum conceito geral granularizando-o em novas proposições, ou seja, novos conceitos e palavra de enlace. Extensão se refere à adição de novos conceitos, porém não a granularização de um conceito anterior.

Um estudo lido foi o "Mapa conceitual: seu potencial como instrumento avaliativo", ele ajudou a entender vantages e desvantagens de um mapa conceitual. Foi realizado com trinta e duas alunas de pedagogia do terceiro semestre que utilizaram o mapa conceitual para sintetizar informações de muitos textos que foram lidos. Todos os mapas eram apresentados para a turma e havia o debate dos mapas em relação a estar abordando a maioria dos conceitos chaves dos textos. As alunas eram identificadas como Ax (sendo x o número de um a trinta e dois, que é a quantidade de participantes). Foram realizados entrevistas, questionários e também solicitado aos grupos que registrassem três vantagens e desvantagens para avaliar a experiência. Vantagens do MC segundo \cite{vantagensDesvantagensMC}: 

\begin{itemize}
\item identificar as dificuldades de aprendizagem
\end{itemize}
A2 diz que "mapas conceituais tornam os conhecimentos mais claros no que se sabe ou não, porque evidencia o que foi aprendido, mostrando também dúvidas, dificuldades e erros" \cite{vantagensDesvantagensMC}.


 \begin{itemize}
\item Favorecer a reelaboração de conceitos a sua consequente sedimentação.
\item Proporcionar feedback quase imediato.
\end{itemize}


 O MC deixa claro a reorganização cognitiva, pois os conceitos conforme são aprofundados e entendidos melhor vão se estendendo e as proposições formadas são alteradas.\cite{vantagensDesvantagensMC}


 \begin{itemize}
\item Integração e ampliação dos conhecimentos.
\end{itemize}
"O trabalho com mapas conceituais nos levou a aprender a identificar os elementos essenciais e inter-relacioná-los" A8 \cite{vantagensDesvantagensMC}.

 Mapas conceituais são ferramentas de avaliações de conhecimento, a partir dele é possível ver onde as pessoas estão errando nos conceitos e nas proposições, também é possível avaliar o quão extenso é a rede de conhecimentos da pessoa em determinado tema. \cite{vantagensDesvantagensMC}.

 Exige capacidade e certo treino e conhecimento para conseguir organizar as informações existentes em seu cérebro de modo a expô-los em conceitos e conectores.
Neste estudo \cite{dificuldadesMapaConceitual} cita a dificuldade de corrigir mapas conceituais comparado a questões de múltipla escolha, por isso diz ser necessário os alunos estarem incluído pois os alunos aprendem mais estando incluídos na correção podendo comparar com os outros ao mesmo tempo com o seu.

Em \cite{vantagensDesvantagensMC} é dito que o mapa conceitual além de ser uma ferramenta avaliativa, também se configurou como estratégia de aprendizagem, vantagem enunciada por 31\% das duplas participantes.

Neste estudo \cite{vantagensDesvantagensMC} que as alunas utilizaram o MC e depois avaliaram o uso, 38\% das alunas declararam que o mapa conceitual possibilita efetivar sucessivas síncreses, análises e sínteses, porém precisa ser discutido em conjunto para sempre aumentar a compreensão.

Deseja-se utilizar o Mapa Conceitual (MC) como ferramenta que avalia o verdadeiro aprendizado, para isso o mapa conceitual precisa ser capaz de evidenciar que existe o conhecimento e a informação presente, e que também as proposições sejam verdadeiras.

Exemplo de trabalho que utilizou mapa conceitual \cite{leiDeNewtonMC} pois está de acordo com \cite{novak} e acredita também que mapas conceituais contribuem para o ensino-aprendizagem.

Mapa conceitual não basta aumentar o tamanho da rede e de ligações, é necessário ter uma análise de conteúdo, se os conceitos estão relacionadas ao tema e se as proposições são verdadeiras ou podem ter sido mal compreendidas \cite{vantagensDesvantagensMC}.

Com os relatos acima do estudo \cite{vantagensDesvantagensMC} citou a respeito das alterações cognitivas que podem acontecem ao se desenhar um MC com proposições verdadeiras. À conceitos e conhecimentos que o cérebro já havia assimilado estes podem se reorganizar tornando a redes de conhecimento do cérebro mais robusta.
