\section[Servidor aprEnDO]{Servidor aprEnDO}
O servidor aprendo hospedado no heroku serve para receber estatísticas dos aplicativos enviados pelos alunos. Os alunos para enviar a estatística precisam informar o número da matrícula. As informações são enviadas em formato JSON e armazenadas no banco de dados integrado com o servidor.

O jogo aprEnDO além do aplicativo de celular tem um servidor para contabilizar dados do jogo.
O servidor está hospedado no heroku utilizando uma contra free que esteve disponível durante a fase de teste dos alunos, apesar de gratuito comparado com a necessidade que era necessário deu para suprir todas as expectativas. Junto com a aplicação foi adicionado um plugin de banco de dados mongodb. Este banco armazenou em um documento os metadados do jogo. O mongodb é um add-on no mLab que permite integração com o heroku, ele permite uma conta temporária com uma quantidade limitada de capacidade de dados, mas para o necessário que são apenas arquivo json. O único problema é que  o mLab não recomenda utilizar a conta free para a produção por não produzir replicação, porém o banco estava sendo acompanhado muitas vezes por dia para coletar os dados recebidos dos alunos. 
O servidor foi escrito em node.js. O script index.js do servidor cria uma rota /estatísticas que é a utilizada pelos aplicativos para enviar a requisição POST. 

O servidor está hospedado no github do heroku, pela interface da linha de comando (CLI) pode ser clonado o projeto com o comando 'heroku git:clone -a servidor-aprendo'.
Para desenvolvimento foi utilizado uma única branch, que foi o necessário. Mais branches seriam criadas, caso fosse necessário. Tendo em vista que o servidor deveria estar pronto até o dia 10 de maio, que era o início da data de aplicação do jogo com os alunos, até esta data o servidor estava sendo desenvolvido e testado, o único problema crucial que poderia ocorrer seria não realizar os commits dos códigos adicionados e removidos. Assim que o servidor estivesse pronto, a expectativa era não precisar alterar mais código nele, apenas se o servidor quebrasse por algum caso inesperado, aí sim o servidor teria de entrar no ar novamente e novas branches de correção seriam criadas para não impedir o funcionamento parcial do servidor. Porém como os logs do servidor estavam sendo acompanhados muitas vezes por dia durante o período de aplicação e o servidor não caiu nenhuma vez, não foi necessário criar branches adicionais para a correção de erros. 

Ao receber os dados json enviados do aplicativo do celular para o servidor, o mesmo concatena todos os dados que chegam em uma variável apenas. Ao terminar a leitura dos dados de chegada eles são formatados é escrito uma resposta de sucesso verdadeiro, a função ‘lê histórico’ é chamada e a resposta é encerrada.

A função ‘lê histórico’ recupera o documento existente no banco de dados heroku\_41w8651l hospedado no mLab, da coleção chamada histórico. A função analisa checa se a matrícula recebida pela requisição já havia alguma vez enviado estatísticas ou se é uma matrícula nova. Em caso de matrícula nova, uma nova linha é adicionada com chave primária a matrícula do aluno e salva no banco. Em caso de matrícula existente é iterado sobre classificação e resolução e todas suas fases e incrementado a quantidade de vitórias a mais que foi concluído das fases no jogo e em seguida o arquivo é atualizado no banco do mLab.
