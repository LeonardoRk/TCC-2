
\section[Brasil com Matemática]{Brasil com Matemática}

Como foi dito na introdução o Brasil não apresenta bons índices de conhecimento de matemática. Um dos porquês desses índices baixos é que existe o desânimo em salas de aula, as vezes por parte dos professores e outras por parte dos alunos. Os professores precisam se reinventar para atrair a atenção dos alunos e melhorar a eficiência do aprendizado em sala de aula. Parte do desânimo dos alunos em sala de aula deve-se por não terem uma base de conteúdo bem solidificada o que pode fazer com que eles não entendam o conteúdo atual e achem a Matemática como algo difícil e chato.

Existem vários exemplos de casos de sucesso, como por exemplo "O bicho papão da matemática virou um gatinho", é um jogo de matemática para alunos do 1º e 2º ano fundamental.O jogo ajuda a fazer divisões. Segundo a notícia no link \url{http://portal.mec.gov.br/component/content/article?id=72701}, 300 alunos se beneficiaram deste projeto. Um dos símbolos que ficou marcado era de uma aluna que tinha reprovado, tirava notas baixas, não interagia muito com os outros alunos e acabou se envolvendo, aprendendo o jogo de tal maneira que passou a tirar 10, ir resolver no quadro e se sentir capaz. É relatado também que não melhorou só em matemática, como em outras matérias. Segundo o professor, além da menina citada, muitos outros que não sabiam divisão aprenderam também.

Mais um estudo \cite{appcalculo}, que é uma proposta de aplicativo gamificado para ensino de cálculo onde é proposto um jogo para o ensino de matemática com o conteúdo voltado para os temas de  conjunto, limite, derivada e integral, não é voltado para o tema de EDO 1ª ordem.

Então até aqui é possível ver que existem muitos jogos no ensino. Porém jogos para matemática no ensino superior não foram encontrados muitos trabalhos, e afunilando um pouco mais para jogos de equação diferencial não foi encontrado nenhum.


O estudo de \cite{revbibmatgam} diz que existem poucos estudos relacionando gamificação com dificuldades em aprendizagem de Matemática, principalmente quando se fala de Matemática no ensino superior. A maioria dos estudos de ensino de Matemática no ensino superior com gamificação são focados para o conteúdo de cálculo 1 (limite, derivada e integral). Nada foi encontrado relacionado ao contexto de gamificação com equações diferenciais e nenhum jogo de equações diferenciais (ED) foi encontrado.

O estudo \cite{revbibmatgam} fez um levantamento bibliográfico sobre gamificação com Matemática e dificuldades no ensino de Matemática e não encontrou nenhum estudo na área de gamificação com dificuldades de aprendizado em Matemática. 
