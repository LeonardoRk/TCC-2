
\section[Brasil com matemática]{Brasil com matemática}
No Brasil 70.3\% dos alunos estão abaixo do nível de conhecimento em matemática. Nível este que de acordo com a Organização para a Cooperação e Desenvolvimento Econômico (OCDE) foi estabelecido para medir a capacidade do alunos em exercer plenamente sua cidadania \cite{inep2015nivelcidadania}. A qualidade do ensino de matemática no Brasil é ruim de acordo com \cite{indiceRuimMat} \cite{inep2015}. O estudo do INEP é realizado a cada 3 anos e é lançado no final do ano seguinte. Foi realizado pela última vez em 2015 quando o Brasil foi 13º colocado em um estudo com 14 países participantes da OCDE. Ficou na frente da República Dominicana e atrás de países como Coréia do Sul, Canadá, Portugal e Estados Unidos. De acordo com o \cite{indiceRuimMat} a posição do Brasil para a qualidade do ensino de matemática e ciências é 133 entre 139 países participantes.

Um dos porquês desses índices baixos é que existe o desânimo em salas de aula, as vezes por parte dos professores e outras por parte dos alunos. Os professores precisam se reinventar para atrair a atenção dos alunos e melhorar a eficiência do aprendizado em sala de aula. Parte do desânimo dos alunos em sala de aula deve-se por achar a matemática como algo chato, não entenderem o conteúdo e não terem uma base de conteúdo bem solidificada.

Outro problema é que existem poucos estudos relacionando gamificação com matemática \cite{revbibmatgam}, principalmente quando se fala de matemática no ensino superior. Quando se encontra matemática para nível superior com gamificação os estudos são focados para o conteúdo de cálculo 1 (limite, derivada e integral). Nada foi encontrado relacionado ao contexto de gamificação + equações diferenciais. Nenhum jogo de equações diferenciais (ED) foi encontrado.

O estudo \cite{revbibmatgam} fez um levantamento bibliográfico sobre gamificação com matemática e dificuldades no ensino de matemática e não encontrou nenhum estudo na área de gamificação com dificuldades de aprendizado em matemática. Porém pelo gráfico \ref{figuramencao} pode indicar que há a dificuldade de aprendizado, já que ocorrem reprovações na matéria e a menção que mais está presente é a MM.

Uma das maneiras de ajudar os alunos a se interessarem mais em sala de aula e atrair a atenção dos mesmos é utilizar o lúdico, ou seja, aprender brincando. Para isso o uso de computadores ou tecnologias da informação como o celular é útil para melhorar o engajamento nas tarefas, principalmente com exercícios e aplicações para a prática das matérias ensinadas em sala de aula \cite{tdahNasEscolas2}.

