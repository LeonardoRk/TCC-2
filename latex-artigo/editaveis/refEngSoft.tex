\section[Engenharia de software]{Engenharia de software}
Hoje em dia softwares estão presente em todos os lugares, tudo que a gente 'toca'. Softwares não se desgastam (físicamente) como hardware, mas estão sujeitos a modificações durante o seu ciclo de vida \cite{Silva_filho}. As modificações as vezes podem causar efeitos acidentais e/ou não esperados. Um software precisa de modificações conforme o tempo passa e para isso acontecer com mais tranquilidade exige a necessidade de uma documentação. Quanto melhor a documentação, mais fácil para entender. Para manter um bom projeto de software é necessário ter ou criar uma cultura de engenharia de software para adotar as melhores práticas. As quais que compreendem os pilares de custo, tempo de desenvolvimento e qualidade de software \cite{Silva_filho}.

No glossário de terminologia de Engenharia de Software da IEEE Std 610.12-1990, define-se Engenharia de Software como a aplicação de uma abordagem sistemática, disciplinada e quantificável para o desenvolvimento, operação e manutenção de um software; isto é aplicação de engenharia de software. Engenharia de software também pode ser o estudo das abordagens \cite{ieeeTerminology}.

Como softwares precisam de manutenção corretiva e/ou evolutiva, também está sujeito a inserção de defeitos decorrentes do desenvolvimento. Estes defeitos podem ser vistos e consertados antes da entrega \cite{Silva_filho} ou ser descoberto pelo usuário que está utilizando e não espera se deparar com o erro. Por isso além de uma documentação, software também precisa de testes, estes que quando bem feitos asseguram a qualidade e confiabilidade do produto de software.

Engenharia de software está presente e tem descrições de melhores práticas em todas as fases desde a concepção, elaboração, construção e transição de um projeto. Seja em um projeto com metologia tradicional ou ágil, um projeto passa por essas fases. A engenharia de software "tem como objetivo apoiar o desenvolvimento profissional de software, cobrindo todos os aspectos da produção de um software."\apud{monsalve}{sucessoJogoEngSoft}

[\cite{sucessoJogoEngSoft}] Benitti e Molléri (2008) concordam que a engenharia de software é uma área muito jovem e sofre contínuas mudanças nos seus fundamentos tecnológicos concretizadas nos métodos e ferramentas de suporte, portanto necessita de métodos de ensino lúdicos e dinâmicos que possam contribuir na aprendizagem do estudante.


Eng de Software percorre o levantamento de requisitos de um jogo, o planejamento e desenvolvimento das funcionalidades, testes para garantir que está tudo funcionando como o occorido e o empacotamento e a entrega para a finalização. Com o tempo também podem precisar de melhorias, manutenção e evoluções. Além de agregar valor para o cliente, também é necessário gerenciar sua infra-estrutura, realizar as configurações para a padronização e fazer o controle de mudanças e gestão da qualidade.

\subsection[Teste de software]{Teste de software}
Teste de software é uma atividade importante do desenvolvimento de software pois está relacionado à qualidade de software, este pode ajudar a verificar o cumprimento dos requisitos.

\cite[p. 17]{Pedro_Henrique} diz que um teste tem basicamente 4 fases, o planejamento, projeto, a execução e a avaliação do resultado dos testes. Já \cite{pressman}, \cite{delamaroJinoMaldonado} dizem que os testes devem acontecer ao longo do processo de desenvolvimento do software, pois é o momento onde as funcionalidades estão frescas no pensamento do desenvolvedor e este deve garantir com testes que ocorra o funcionamento esperando das funções.

