\section[Engenharia de software]{Engenharia de software}
Engenharia de software está presente e tem melhores práticas em todas as fases desde a concepção, elaboração, construção e transição de um projeto. Seja em um projeto com metologia tradicional ou ágil, um projeto passa por essas fases.
"Tem como objetivo apoiar o desenvolvimento profissional de software, cobrindo todos os aspectos da produção de um software."[\cite{sucessoJogoEngSoft}  Monsalve et al. (2010)]

[\cite{sucessoJogoEngSoft}] Benitti e Molléri (2008) concordam que a engenharia de software é uma área muito jovem e sofre contínuas mudanças nos seus fundamentos tecnológicos concretizadas nos métodos e ferramentas de suporte, portanto necessita de métodos de ensino lúdicos e dinâmicos que possam contribuir na aprendizagem do estudante



Eng de Software percorre o levantamento de requisitos de um jogo, o planejamento e desenvolvimento das funcionalidades, testes para garantir que está tudo funcionando como o occorido e o empacotamente e a entrega para a finalização. Com o tempo também podem precisar de melhorias, manutenção e evoluções. O mundo dev-ops é onde o desenvolvedor além de agregar valor para o cliente, também gerencia sua infra-estrutura, realiza as configurações para a padronização e faz o controle de mudanças e gestão da qualidade.
