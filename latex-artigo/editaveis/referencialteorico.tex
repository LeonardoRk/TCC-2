\chapter[Referencial Teórico]{Referencial Teórico}
Existem estudos com referencial teórico apoiando o uso de gamificação em vários contextos e utilizando dos benefícios oferecidos por ela. Exemplo de contextos como marketing, saúde, educação e etc. A gamificação é também utilizada em contextos de ensino de matemática (embora não sejam muitos os resultados encontrados para universidades). Em alguns casos são utilizados a tecnologia junto da gamificação nos contextos acima citado. É dito que a tecnologia e jogos contribuem para ajudar no engajamento e chamar atenção das pessoas e dos estudantes.
Tem um estudo dizendo que não há limites para a idade de jogar, brincar e se divertir. Por isso jogo e gamificação podem ser possíveis estratégias para utilizar em comunhão de tecnologias visando o ensino de EDO 1ª ordem para estudantes do ensino superior.

\section[Mapa Conceitual (MC)  como ferramenta de avaliação]{Mapa Conceitual (MC) como ferramenta de avaliação}

Um estudo lido foi o "Mapa conceitual: seu potencial como instrumento avaliativo", ele ajudou a entender vantages e desvantagens de um mapa conceitual, através de 32 alunas de pedagogia do 3º que utilizaram o mapa conceitual para sintetizar informações de textos que foram lidos como forma de avaliação. Algumas observações identificadas foram: 
O relato de uma das estudantes que utilizou o mapa "O trabalho com mapas conceituais nos levou a aprender a identificar os elementos essenciais e inter-relacioná-los" A8. A8 é a forma de anonimato das estudantes em \cite{vantagensDesvantagensMC}.

Outro relato, porém agora de A2 é "mapas conceituais tornam os conhecimentos mais claros no que se sabe ou não, porque evidencia o que foi aprendido, mostrando também dúvidas, dificuldades e erros" \cite{vantagensDesvantagensMC}.

Vantagens dos MC segundo \cite{vantagensDesvantagensMC}: 
\begin{itemize}
\item[]identificar as dificuldades de aprendizagem
\item[]favorecer a reelaboração de conceitos a sua consequente sedimentação
\item[]integração e ampliação dos conhecimentos
\item[]proporcionar feedback quase imediato
\end{itemize}

Em \cite{vantagensDesvantagensMC} é dito que o mapa conceitual além de ser uma ferramenta avaliativa, também se configurou como estratégia de aprendizagem, vantagem enunciada por 31\% das duplas participantes.

Exemplo de trabalho que utilizou mapa conceitual \cite{leiDeNewtonMC} pois está de acordo com \cite{novak} e acredita também que mapas conceituais contribuem para o ensino-aprendizagem. 

Outro exemplo de estudo que utilizou o mapa conceitual, este cita a dificuldade de corrigir mapas conceituais comparado a questões de múltipla escolha, por isso diz ser necessário os alunos estarem incluído pois os alunos aprendem mais estando incluídos na correção podendo comparar com os outros ao mesmo tempo com o seu \cite{dificuldadesMapaConceitual}.

Mapa conceitual é uma estrutura hierárquica, iniciados por conceitos mais abrangentes, os quais progressivamente vão sendo relacionados com conceitos mais específicos e esclarecendo suas relações de subordinação \apud{novak}{vantagensDesvantagensMC}

Mapas conceituais são ferramentas de avaliações de conhecimento, a partir dele é possível ver onde as pessoas estão errando nos conceitos e nas proposições, também é possível avaliar o quão extenso é a rede de conhecimentos da pessoa em determinado assunto. Produzir um mapa conceitual antes e depois de aprender um conteúdo fica registrado no próprio mapa o quanto evoluiu o conhecimento da pessoa \cite{vantagensDesvantagensMC}.

Mapa conceitual não basta aumentar o tamanho da rede e de ligações, é necessário ter uma análise de conteúdo, se os conceitos estão relacionadas ao tema e se as proposições são verdadeiras ou podem ter sido mal compreendidas \cite{vantagensDesvantagensMC}.

"Apesar de a aprendizagem implicar a elaboração e a reelaboração do conhecimento pelo educando, ela também permanece refém de interações com os pares e com o professor" \cite[p. 180]{vantagensDesvantagensMC}

Neste estudo \cite{vantagensDesvantagensMC} que as alunas utilizaram o MC e depois avaliaram o uso, 38\% das alunas declararam que o mapa conceitual possibilita efetivar sucessivas síncreses, análises e sínteses, porém precisa ser discutido em conjunto para sempre aumentar a compreensão.

O mapa conceitual deixa claro a reorganização cognitiva, pois os conceitos conforme são aprofundados e entendidos melhor vão se estendendo e as proposições formadas são alteradas.\cite{vantagensDesvantagensMC}



\section[Sucesso de jogos no ensino]{Sucesso de jogos no ensino}

Este link é uma notícia no portal.mec do sucesso do jogo de matemática para alunos do 1º e 2º ano fundamental.O jogo ajuda a fazer divisões, segundo a notícia 300 alunos se beneficiaram deste projeto que se chama "O bicho papão da matemática virou um gatinho". Um dos símbolos que ficou marcado era de uma aluna que tinha reprovado, tirava notas baixas, não interagia muito com os outros alunos, acabou se envolvendo e aprendendo o jogo de tal maneira que passou a tirar 10, ir resolver no quadro e se sentir capaz. Relata também que não melhorou só em matemática, como em outras matérias. Segundo o professor, além da menina citada, muitos outros que não sabiam divisão aprenderam também. (http://portal.mec.gov.br/component/content/article?id=72701)

Existem muitos jogos sendo utilizados em contextos educacionais, apenas neste estudo \cite{sucessoJogoEngSoft} que fez um levantamento de jogos específicos para o uso na engenharia de software é possível encontrar a listagem de 20 jogos no apêndice. Porém jogos para matemática no ensino superior não foram encontrados muitos trabalhos, e filtrando um pouco mais para jogos de equação diferencial não foi encontrado nenhum.


Alguns jogos tem sucesso como meio de ensino e em várias outras áreas como por exemplo engenharia de software, marketing entre outras[] e em colégio para crianças[artigo paraná, notícia de natal do MEC]

Atratividade de jogos está relacionada a mecanismos psicológicos e sociais \cite{sucessoJogoEngSoft}

Para o jogo ser bem aceito e cumprir com a sua meta, ele deve dar uma boa base de conhecimento e motivação \cite{sucessoJogoEngSoft}.

"Uma das propostas de melhoria de aprendizado em sala de aula são os jogos educacionais" \cite[p. 4]{sucessoJogoEngSoft}

"Na literatura, encontram-se vários trabalhos que demonstram profissionais de educação utilizando-os como ferramenta de auxílio ao aprendizado" \cite[p. 3]{sucessoJogoEngSoft}

"Como a finalidade na sala de aula é estimular ideias dos alunos, ensinar apenas com aulas expositivas
tradicionais pode dificultar o aprendizado." \cite[p. 4]{sucessoJogoEngSoft} 

"Uma das propostas metodológicas para ensino de engenharia de software e suas
disciplinas, são os jogos educacionais. Sabe-se que os jogos educacionais, segundo
Nunes e Parreira (2015), têm sido intensamente utilizados por profissionais da área de
educação como auxílio para a construção do conhecimento. Em sua pesquisa, Fukusawa
et al. (2015) apontam alguns dos benefícios que os jogos educacionais podem trazer ao
processo de ensino e aprendizagem como, por exemplo, a motivação e o aprendizado
por descoberta. Portanto, os jogos podem proporcionar a vivência em experiências de
aprendizagem concretas [Monsalve et al. 2010]." \cite[4]{sucessoJogoEngSoft}

Silva et al. (2015) comentam que uma abordagem alternativa às aulas tradicionais, devido elas
serem mais teóricas e expositivas, é a utilização dos jogos, pois esta abordagem preza
por uma teoria de motivação humana como ponto de partida. \cite{sucessoJogoEngSoft} 

\begin{citacao}
Foram pesquisados nos Anais da base WEI e selecionados os artigos que tivessem trabalhos com jogos que auxiliassem no aprendizado de ensino superior das disciplinas de Engenharia de Software, referencialmente os que validassem com alunos. \cite{sucessoJogoEngSoft}
\end{citacao}


\section[Engenharia de software]{Engenharia de software}
Engenharia de software está presente e tem melhores práticas em todas as fases desde a concepção, elaboração, construção e transição de um projeto. Seja em um projeto com metologia tradicional ou ágil, um projeto passa por essas fases.
"Tem como objetivo apoiar o desenvolvimento profissional de software, cobrindo todos os aspectos da produção de um software."[\cite{sucessoJogoEngSoft}  Monsalve et al. (2010)]

[\cite{sucessoJogoEngSoft}] Benitti e Molléri (2008) concordam que a engenharia de software é uma área muito jovem e sofre contínuas mudanças nos seus fundamentos tecnológicos concretizadas nos métodos e ferramentas de suporte, portanto necessita de métodos de ensino lúdicos e dinâmicos que possam contribuir na aprendizagem do estudante



Eng de Software percorre o levantamento de requisitos de um jogo, o planejamento e desenvolvimento das funcionalidades, testes para garantir que está tudo funcionando como o occorido e o empacotamente e a entrega para a finalização. Com o tempo também podem precisar de melhorias, manutenção e evoluções. O mundo dev-ops é onde o desenvolvedor além de agregar valor para o cliente, também gerencia sua infra-estrutura, realiza as configurações para a padronização e faz o controle de mudanças e gestão da qualidade.




\section[Problemas com matemática no Brasil]{Problemas com matemática no Brasil}
No Brasil 70.3\% dos alunos estão abaixo do nível de conhecimento em matemática. Nível este que de acordo com a Organização para a Cooperação e Desenvolvimento Econômico (OCDE) foi estabelecido para medir a capacidade do alunos em exercer plenamente sua cidadania \cite{inep2015nivelcidadania}. A qualidade do ensino de matemática no Brasil é ruim de acordo com \cite{indiceRuimMat} \cite{inep2015}. O estudo do INEP é realizado a cada 3 anos e é lançado no final do ano seguinte. Foi realizado pela última vez em 2015 quando o Brasil foi 13º colocado em um estudo com 14 países participantes da OCDE. Ficou na frente da República Dominicana e atrás de países como Coréia do Sul, Canadá, Portugal e Estados Unidos. De acordo com o \cite{indiceRuimMat} a posição do Brasil para a qualidade do ensino de matemática e ciências é 133 entre 139 países participantes.

Um dos porquês desses índices baixos é que existe o desânimo em salas de aula, as vezes por parte dos professores e outras por parte dos alunos. Os professores precisam se reinventar para atrair a atenção dos alunos e melhorar a eficiência do aprendizado em sala de aula. Parte do desânimo dos alunos em sala de aula deve-se por achar a matemática como algo chato, não entenderem o conteúdo e não terem uma base de conteúdo bem solidificada.

Outro problema é que existem poucos estudos relacionando gamificação com matemática \cite{revbibmatgam}, principalmente quando se fala de matemática no ensino superior. Quando se encontra matemática para nível superior com gamificação os estudos são focados para o conteúdo de cálculo 1 (limite, derivada e integral). Nada foi encontrado relacionado ao contexto de gamificação + equações diferenciais. Nenhum jogo de equações diferenciais (ED) foi encontrado.

O estudo \cite{revbibmatgam} fez um levantamento bibliográfico sobre gamificação com matemática e dificuldades no ensino de matemática e não encontrou nenhum estudo na área de gamificação com dificuldades de aprendizado em matemática. Porém pelo gráfico \ref{figuramencao} pode indicar que há a dificuldade de aprendizado, já que ocorrem reprovações na matéria e a menção que mais está presente é a MM.

Uma das maneiras de ajudar os alunos a se interessarem mais em sala de aula e atrair a atenção dos mesmos é utilizar o lúdico, ou seja, aprender brincando. Para isso o uso de computadores ou tecnologias da informação como o celular é útil para melhorar o engajamento nas tarefas, principalmente com exercícios e aplicações para a prática das matérias ensinadas em sala de aula \cite{tdahNasEscolas2}.


\section[Ajuda de jogos para ensino de matemática]{Ajuda de jogos para ensino de matemática}
Jogo é prática que ajuda na concretização do conhecimento, além de tornar o ambiente mais prazeroso \cite{jogoPratPedagoc}. É importante avaliar se as pessoas estão se divertindo no momento de aprendizado, já que brincar contribui na formação do estudante, tanto social quanto intelectual \cite{jogoPratPedagoc}.

"O caráter lúdico, bem como a possibilidade de atuação crítica, proporciona ao aluno uma participação efetiva no processo de ensino aprendizagem, se tornando um momento ímpar de crescimento pessoal e coletivo."  \cite{jogoPratPedagoc}. O que significa que contribui para o aluno se tornar um ser ativo e pensante, capacitando-o a exercer seu papel como cidadão.

"Os jogos despertam o interesse dos jovens trazendo diversos benefícios aliados à
educação[...]" \cite{appcalculo}

\begin{comment}
Gamificação foca em elementos como desafios, níveis, avatar, conquistas, histórias, pontos (Gustavo Fortes Tondello, PhD). Esses elementos são utilizados para engajamento do jogador.

Completar missões e derrotar um chefão faz o jogador se sentir competente (Gustavo Fortes Tondello, PhD).
Ser capaz de escolher diferentes caminhos ou criar coisas diferentes faz o jogador se sentir autônomo (Gustavo Fortes Tondello, PhD).
\end{comment}

Tem também o estudo de \cite{appcalculo}, que é uma proposta de aplicativo gamificado para ensino de cálculo onde é proposto um jogo para o ensino de matemática, porém o conteúdo abordado é de conjunto, limite, derivada e integral, não é voltado para o tema de EDO 1ª ordem.

No estudo \cite{revbibmatgam} foi feito uma revisão bibliográfica nas bases de dados Scielo, Science Direct, ACM Library, IEE Xplore Digital entre outras, para levantar o uso de gamificação e dificuldades matemáticas. De 2008 trabalhos, nenhum eram relacionando gamificação e dificuldades de matemática. No entando para \cite{dicheva} que é citado no estudo, a falta de pesquisa na área é justificada por ser uma temática nova.

As referências citadas neste tópico reforçam que jogos podem ser utilizados para fazer os estudantes gostarem e se atreverem mais no contexto da matemática. Espera-se que os alunos busquem e tenham a vontade do conhecimento por si próprio para que se tornem mais independentes. Também conclui-se que existem poucos estudos na área de C2, específico para ED, apesar de terem estudos na área de matemática, estes destinam-se a cálculo 1 e matérias do ensino fundamental.

\section[Contribuição de jogos para ensino inclusivo]{Contribuição de jogos para ensino inclusivo}

O trecho a seguir mostra o lúdico como um facilitador para inserir alunos com déficits em uma sociedade. 

"[...] a percepção do lúdico passa a ser vista como uma aliada para os professores no que tange a orientar os alunos portadores de necessidades especiais em busca do desenvolvimento das suas habilidades e potencialidades dentro de uma perspectiva inclusiva"  \cite{jogoPratPedagoc}.

De acordo com \cite{jogoPratPedagoc} o jogo é um benefício no ambiente inclusivo pois ajuda a adquirir conhecimento.

É pretendido utilizar a teconologia como ferramenta de aprendizado lúdico. O jogo é importante e eficaz para escolas inclusivas \cite{jogoPratPedagoc}.

\begin{comment}
Foram anotadas algumas observações sobre os experimentos realizados em estudantes com TDAH para ser lembrado de incluir nos requisitos do jogo para aumentar o público alvo. A seguir alguns itens: 

Crianças com TDAH precisam de um feedback mais frequente. O feedback deve ser apresentado no momento do comportamento visado. Porém o próprio feedback pode causar uma distração, então ele deve ser feito seguido de uma chamada de atenção que redirecione o aluno para a próxima tarefa a ser cumprida (George J. DuPaul, PhD e Gary Stoner, PhD. p. 131-132).

O nível de complexidade das equações e das dificuldades aumentará gradativamente. Para pessoas com TDAH a facilidade em jogar é maior quando inicia-se com instruções iniciais simples e com pouco número de etapas. Para fixar melhor o entendimento do aluno sobre a atividade pede-se que a pessoa repita o que foi pedido como objetivo para completar a atividade. Ao existirem erros em atividade é bom variar as atividades para evitar o cansaço por repetição (George J. DuPaul, PhD e Gary Stoner, PhD. p.132).

Recompensas devem ser dadas para que a motivação aumente. Porém a pessoa deve escolher o prêmio desejado ao invés de ser entregue um prêmio que ela não deseje. Para isso antes da atividade é necessário a fase de negociação para que seja escolhidos possíveis prêmios que sirvam como motivação (George J. DuPaul, PhD e Gary Stoner, PhD. p.133).

\end{comment}


\section[Suporte da tecnologia com jogos nos colégios]{Suporte da tecnologia com jogos nos colégios}
Queremos fortalecer que gamificação também pode ser combinada para ajudar a incluir os alunos com dificuldades especiais na faculdades e no meio social. "[...] a tecnologia é uma grande aliada para a aprendizagem dos alunos portadores do TDAH."  \cite{matEtdah1}.

O computador e tecnologias como celular, além de serem ferramentas de auxílio, são também  motivadoras para os estudantes \cite{softwaregamificado}.

É reforçado em \cite{revbibmatgam} que a tecnologia deve ser usada como ferramenta auxiliadora para o ensino de matemática em colégios inclusivos.

Este tópico visa reafirmar o potencial dos jogos tecnológicos para ser usado como ferramenta de atração dos alunos para os colégios e universidades.