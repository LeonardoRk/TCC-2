\chapter[Referencial Teórico]{Referencial Teórico}
O referencial teórico presente discorrerá a respeito de quatro temas para defender a ideia deste trabalho. O primeiro tema contextualiza o Brasil com a matemática, depois abordaremos o uso de jogos como um facilitador no ensino, seja para jogos de matemática ou não, sejam jogos eletrônicos ou não. O terceiro tema justificará o uso do mapa conceitual como uma ferramenta de validação de conhecimento adquirido e por fim trará conceitos da engenharia de software para um bom processo de desenvolvimento de software em jogos e/ou aplicações gamificadas. 


\section[Brasil com Matemática]{Brasil com Matemática}
No Brasil 70.3\% dos alunos estão abaixo do nível de conhecimento em Matemática. Nível este que de acordo com a Organização para a Cooperação e Desenvolvimento Econômico (OCDE) foi estabelecido para medir a capacidade do alunos em exercer plenamente sua cidadania \cite{inep2015nivelcidadania}. A qualidade do ensino de Matemática no Brasil é ruim de acordo com \cite{inep2015}. O estudo do INEP é realizado a cada três anos e é lançado no final do ano seguinte. Foi realizado pela última vez em 2015 quando o Brasil foi 13º colocado em um estudo com 14 países participantes da OCDE. Ficou na frente da República Dominicana e atrás de países como Coréia do Sul, Canadá, Portugal e Estados Unidos. De acordo com o \cite{indiceRuimMat} a posição do Brasil para a qualidade do ensino de Matemática e ciências é 133 entre 139 países participantes.

Um dos porquês desses índices baixos é que existe o desânimo em salas de aula, as vezes por parte dos professores e outras por parte dos alunos. Os professores precisam se reinventar para atrair a atenção dos alunos e melhorar a eficiência do aprendizado em sala de aula. Parte do desânimo dos alunos em sala de aula deve-se por achar a Matemática como algo chato, não entenderem o conteúdo e não terem uma base de conteúdo bem solidificada.

Outro problema é que existem poucos estudos relacionando gamificação com Matemática \cite{revbibmatgam}, principalmente quando se fala de Matemática no ensino superior. Quando se encontra Matemática para nível superior com gamificação os estudos são focados para o conteúdo de cálculo 1 (limite, derivada e integral). Nada foi encontrado relacionado ao contexto de gamificação com equações diferenciais. Nenhum jogo de equações diferenciais (ED) foi encontrado.

O estudo \cite{revbibmatgam} fez um levantamento bibliográfico sobre gamificação com Matemática e dificuldades no ensino de Matemática e não encontrou nenhum estudo na área de gamificação com dificuldades de aprendizado em Matemática. 

Uma das maneiras de ajudar os alunos a se interessarem mais em sala de aula e atrair a atenção dos mesmos é utilizar o lúdico, ou seja, aprender brincando. Para isso o uso de computadores ou tecnologias da informação como o celular é útil para melhorar o engajamento nas tarefas, principalmente com exercícios e aplicações para a prática das matérias ensinadas em sala de aula \cite{tdahNasEscolas2}.


\section[Sucesso de jogos no ensino]{Sucesso de jogos no ensino}

Alguns autores diferenciam jogos e gamificação, porém neste estudo ambos serão tratados de maneira igual querendo se referir à atividades lúdicas. Esta seção visa apoiar jogos como uma estratégia boa para aprendizado dos alunos/jogadores e também como uma boa ferramenta para ser utilizada como aprendizagem. Espera-se que com o uso de jogos os estudantes gostem e se atrevam mais no contexto da Matemática buscando por conta própria o conhecimento e que se tornem mais independentes.

No século XXI com todo o avanço tecnológico o celular se tornou um aparelho indispensável para o ser humano, então nada melhor do que unir o útil ao agradável e utilizá-lo como uma ferramenta de auxílio e de motivação para os estudantes \cite{softwaregamificado}. O uso de computadores ou tecnologias da informação como o celular é útil para melhorar o engajamento nas tarefas, principalmente com exercícios e aplicações para a prática das matérias ensinadas em sala de aula \cite{tdahNasEscolas2}.

Como a finalidade na sala de aula é estimular ideias dos alunos, ensinar apenas com aulas expositivas tradicionais pode dificultar o aprendizado. Por isso os jogos chegaram às salas de aula para ajudar \cite[p. 4]{sucessoJogoEngSoft}. De acordo com o dicionário Michaelis, uma das definições de jogo é "Qualquer atividade recreativa que tem por finalidade entreter, divertir ou distrair" \cite{Michaelis}. Alguns sinônimos são: brincadeira, entretenimento, diversão, entre outros. O jogo tem além do seu papel lúdico, a característica da construção de conhecimento, ou seja, fazer com que o jogador aprenda através das experiências adquiridas por ele \cite{jogoPratPedagoc} \cite{appcalculo} \cite{Nunes} \cite{fukusawa}. De acordo com \cite{jogoPratPedagoc}, o caráter lúdico proporciona ao aluno uma participação efetiva no processo de ensino aprendizagem se tornando um momento ímpar de crescimento pessoal e coletivo. 

Os jogos e a gamificação se mostraram como uma estratégia efetiva tanto no ensino fundamental quanto no ensino superior. Na literatura, encontram-se vários trabalhos que demonstram profissionais de educação utilizando os jogos como ferramenta de auxílio ao aprendizado \cite[p. 3]{sucessoJogoEngSoft}. Abaixo seguem alguns exemplos.

\begin{itemize}
  \item Um jogo chamado "O bicho papão da Matemática virou um gatinho" que ensina Matemática para alunos do 1º e 2º ano fundamental. Este ajuda os alunos a fazer divisões. Segundo a notícia disponível no \href{http://portal.mec.gov.br/component/content/article?id=72701}{portal mec}, 300 alunos se beneficiaram deste projeto. Um dos símbolos que ficou marcado era de uma aluna que tinha reprovado, tirava notas baixas, não interagia muito com os outros alunos e acabou se envolvendo, aprendendo o jogo de tal maneira que passou a tirar 10, ir resolver no quadro e se sentir capaz. É relatado também que não melhorou só em Matemática, como em outras matérias. Segundo o professor, além da menina citada, muitos outros alunos que não sabiam divisão aprenderam também.


  \item O estudo \cite{sucessoJogoEngSoft} realizou uma pesquisa nos Anais da base WEI para selecionar artigos que contém jogos que auxiliam no aprendizado das disciplinas de Engenharia de Software. Foram selecionados preferencialmente os estudos que validassem os resultados com alunos \cite{sucessoJogoEngSoft}. Como resultado foi obtido um levantamento de 20 jogos focados em disciplinas do curso, dentre elas gestão de projetos, algoritmos, estrutura de dados, teste de software entre outras.

  \item O estudo \cite{appcalculo} é uma proposta de aplicativo gamificado para ensino de Cálculo 1, onde é proposto um jogo para o ensino de Matemática com o conteúdo voltado para os temas de conjunto, limite, derivada e integral.

  \item O estudo de \cite{jogoSuporteMat} cita um jogo educacional de matemática com um agente que ensina aritmética que é capaz de aprender. O programa utiliza de inteligência artificial.   
\end{itemize}


Alguns dos motivos pelo qual os profissionais da educação optam pela utilização de jogos são:

\begin{itemize}
	\item Despertar o interesse dos jovens e trazer diversos benefícios à educação \cite{appcalculo}.
	\item O jogo é uma prática que ajuda na concretização do conhecimento, além de tornar o ambiente mais prazeroso \cite{jogoPratPedagoc}. 
	\item Motivação e aprendizado por descoberta \apud[p. 2]{fukusawa}{savi}.
	\item Desenvolvimento de habilidades cognitivas, experiência de novas identidades, socialização, coordenação motora e comportamento expert \cite[p. 3 e 4]{savi}.
	\item Poder proporcionar a vivência em experiências de aprendizagem concretas \apud[p. 4]{monsalve}{sucessoJogoEngSoft}.
\end{itemize}

De acordo com \cite{Nunes}, sabe-se que os jogos educacionais têm sido intensamente utilizados por profissionais da área de educação como auxílio para a construção do conhecimento e para motivar os alunos, porém os resultados encontrados mostraram que apesar de existirem muitos jogos, poucos são de Matemática. Dos jogos de Matemática encontrados, foi constatado que uma minoria deles são para o ensino superior, e dos jogos de Matemática para o ensino superior apenas um deles falou de equações diferenciais, que é o estudo \cite{videoGameED} o qual aplica os conceitos na movimentação dos personagens (princípio da dinâmica de Newton), porém o jogo não é focado no ensino de equação diferencial ordinária.

O estudo de \cite{revbibmatgam} é uma revisão sistemática de literatura realizada nas bases de dados Scielo Library, BIREME Biblioteca, Science Direct, ACM Library e IEEE Xplore Digital Library e nos periódicos Revista Brasileira de Informática na Educação e a Revista de Novas Tecnologias na Educação. O estudo procurou artigos que constatavam a existência de ferramentas relacionadas com gamificação e dificuldades de aprendizagem de matemática. De 2008 trabalhos, nenhum eram relacionando gamificação e dificuldades de matemática. Por fim o estudo concluiu que:

\begin{citacao}
identifica-se a necessidade de pesquisas sobre esta temática, já que as dificuldades de aprendizagem na Matemática são frequentes em sala de aula, e a gamificação tem-se mostrado uma ferramenta promissora nos ambientes de ensino e aprendizagem em todos os níveis de ensino. 
\end{citacao}

No entanto para \cite{dicheva} que é citado no estudo \cite{revbibmatgam}, a falta de pesquisa nesta área é justificada por ser uma temática nova. Acredita-se que com o tempo o número de interessados e pesquisas na área aumentará.

Com todas as constatações acima, espera-se ter convencido que muitos autores já estão utilizando jogos ou aplicações como uma ferramenta de auxílio para o ensino e que em alguns casos elas são um sucesso, fazendo com que de fato os jogadores aprendam o conteúdo que o jogo deseja transmitir. Constatando que há poucos jogos de Matemática para equações diferenciais este trabalho trata-se do desenvolvimento de um jogo para o ensino de matemática à alunos do ensino superior.


\section[Engenharia de software]{Engenharia de software}
Hoje em dia softwares estão presente em todos os lugares, tudo que a gente 'toca'. Softwares não se desgastam (físicamente) como hardware, mas estão sujeitos a modificações durante o seu ciclo de vida \cite{Silva_filho}. As modificações as vezes podem causar efeitos acidentais e/ou não esperados. Um software precisa de modificações conforme o tempo passa e para isso acontecer com mais tranquilidade exige a necessidade de uma documentação. Quanto melhor a documentação, mais fácil para entender. Para manter um bom projeto de software é necessário ter ou criar uma cultura de engenharia de software para adotar as melhores práticas. As quais que compreendem os pilares de custo, tempo de desenvolvimento e qualidade de software \cite{Silva_filho}.

No glossário de terminologia de Engenharia de Software da IEEE Std 610.12-1990, define-se Engenharia de Software como a aplicação de uma abordagem sistemática, disciplinada e quantificável para o desenvolvimento, operação e manutenção de um software; isto é aplicação de engenharia de software. Engenharia de software também pode ser o estudo das abordagens \cite{ieeeTerminology}.

Como softwares precisam de manutenção corretiva e/ou evolutiva, também está sujeito a inserção de defeitos decorrentes do desenvolvimento. Estes defeitos podem ser vistos e consertados antes da entrega \cite{Silva_filho} ou ser descoberto pelo usuário que está utilizando e não espera se deparar com o erro. Por isso além de uma documentação, software também precisa de testes, estes que quando bem feitos asseguram a qualidade e confiabilidade do produto de software.

Engenharia de software está presente e tem descrições de melhores práticas em todas as fases desde a concepção, elaboração, construção e transição de um projeto. Seja em um projeto com metologia tradicional ou ágil, um projeto passa por essas fases. A engenharia de software "tem como objetivo apoiar o desenvolvimento profissional de software, cobrindo todos os aspectos da produção de um software."\apud{monsalve}{sucessoJogoEngSoft}

[\cite{sucessoJogoEngSoft}] Benitti e Molléri (2008) concordam que a engenharia de software é uma área muito jovem e sofre contínuas mudanças nos seus fundamentos tecnológicos concretizadas nos métodos e ferramentas de suporte, portanto necessita de métodos de ensino lúdicos e dinâmicos que possam contribuir na aprendizagem do estudante.


Eng de Software percorre o levantamento de requisitos de um jogo, o planejamento e desenvolvimento das funcionalidades, testes para garantir que está tudo funcionando como o occorido e o empacotamento e a entrega para a finalização. Com o tempo também podem precisar de melhorias, manutenção e evoluções. Além de agregar valor para o cliente, também é necessário gerenciar sua infra-estrutura, realizar as configurações para a padronização e fazer o controle de mudanças e gestão da qualidade.

\subsection[Teste de software]{Teste de software}
Teste de software é uma atividade importante do desenvolvimento de software pois está relacionado à qualidade de software, este pode ajudar a verificar o cumprimento dos requisitos.

\cite[p. 17]{Pedro_Henrique} diz que um teste tem basicamente 4 fases, o planejamento, projeto, a execução e a avaliação do resultado dos testes. Já \cite{pressman}, \cite{delamaroJinoMaldonado} dizem que os testes devem acontecer ao longo do processo de desenvolvimento do software, pois é o momento onde as funcionalidades estão frescas no pensamento do desenvolvedor e este deve garantir com testes que ocorra o funcionamento esperando das funções.



\begin{comment}
Existem estudos com referencial teórico apoiando o uso de gamificação em vários contextos e utilizando dos benefícios oferecidos por ela. Exemplo de contextos como marketing, saúde, educação e etc. A gamificação é também utilizada em contextos de ensino de matemática (embora não sejam muitos os resultados encontrados para universidades). Em alguns casos são utilizados a tecnologia junto da gamificação nos contextos acima citado. É dito que a tecnologia e jogos contribuem para ajudar no engajamento e chamar atenção das pessoas e dos estudantes.
Tem um estudo dizendo que não há limites para a idade de jogar, brincar e se divertir. Por isso jogo e gamificação podem ser possíveis estratégias para utilizar em comunhão de tecnologias visando o ensino de EDO 1ª ordem para estudantes do ensino superior.
 
 \end{comment}
