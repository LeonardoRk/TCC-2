\chapter[Referencial Teórico]{Referencial Teórico}
Existem estudos com referencial teórico apoiando o uso de gamificação em vários contextos e utilizando dos benefícios oferecidos por ela. Exemplo de contextos como marketing, saúde, educação e etc. A gamificação é também utilizada em contextos de ensino de matemática (embora não sejam muitos os resultados encontrados para universidades), e para escolas inclusivas. Em alguns casos são utilizados a tecnologia junto da gamificação nos contextos acima citado. É dito que a tecnologia e jogos contribuem para ajudar no engajamento e chamar atenção das pessoas e dos estudantes.
Tem um estudo dizendo que não há limites para a idade de jogar, brincar e se divertir. Por isso jogo e gamificação podem ser possíveis estratégias para utilizar em comunhão de tecnologias visando o ensino de EDO 1ª ordem para estudantes do ensino superior.

\section[Problemas com matemática no Brasil]{Problemas com matemática no Brasil}
No Brasil 70.3\% dos alunos estão abaixo do nível de conhecimento em matemática. Nível esse que de acordo com a Organização para a Cooperação e Desenvolvimento Econômico (OCDE) foi estabelecido para medir a capacidade do alunos em exercer plenamente sua cidadania \cite{inep2015nivelcidadania}. A qualidade do ensino de matemática no Brasil é ruim de acordo com \cite{indiceRuimMat} \cite{inep2015}. O estudo do INEP é realizado a cada 3 anos e é lançado no final do ano seguinte. Foi realizado pela última vez em 2015 quando o Brasil foi 13º colocado em um estudo com 14 países participantes da OCDE. Ficou na frente da República Dominicana e atrás de países como Coréia do Sul, Canadá, Portugal e Estados Unidos. De acordo com o \cite{indiceRuimMat} a posição do Brasil para a qualidade do ensino de matemática e ciências é 133 entre 139 países participantes.

O porquê desses índices baixos é que existe o desânimo em salas de aula, as vezes por parte dos professores e outras por parte dos alunos. Os professores precisam se reinventar para atrair a atenção dos alunos e melhorar a eficiência do aprendizado em sala de aula. Parte do desânimo dos alunos em sala de aula deve-se por achar a matemática como algo chato, não entenderem o conteúdo e não terem uma base de conteúdo bem solidificada.

Outro problema é que existem poucos estudos relacionando gamificação com matemática \cite{revbibmatgam}, principalmente quando se fala de matemática no ensino superior. Quando se encontra matemática para nível superior com gamificação os estudos são focados para o conteúdo de cálculo 1 (limite, derivada e integral). Nada foi encontrado relacionado ao contexto de gamificação + equações diferenciais. Nenhum jogo de equações diferenciais (ED) foi encontrado.

O estudo \cite{revbibmatgam} fez um levantamento bibliográfico sobre gamificação com matemática e dificuldades no ensino de matemática e não encontrou nenhum estudo na área de gamificação com dificuldades de aprendizado em matemática. Porém pelo gráfico \ref{figuramencao} pode indicar que há a dificuldade de aprendizado, já que ocorrem reprovações na matéria e a menção que mais está presente é a MM.

Uma das maneiras de ajudar os alunos a se interessarem mais em sala de aula e atrair a atenção dos mesmos é utilizar do lúdico, ou seja, aprender brincando. Para isso o uso de computadores ou tecnologias da informação como o celular é útil para melhorar o engajamento nas tarefas, principalmente com exercícios e aplicações para a prática das matérias ensinadas em sala de aula \cite{tdahNasEscolas2}.


\section[Ajuda de jogos para ensino de matemática]{Ajuda de jogos para ensino de matemática}
Jogo é prática que ajuda na concretização do conhecimento, além de tornar o ambiente mais prazeroso \cite{jogoPratPedagoc}. É importante avaliar se as pessoas estão se divertindo no momento de aprendizado, já que brincar contribui na formação do estudante, tanto social quanto intelectual \cite{jogoPratPedagoc}.

"O caráter lúdico, bem como a possibilidade de atuação crítica, proporciona ao aluno uma participação efetiva no processo de ensino aprendizagem, se tornando um momento ímpar de crescimento pessoal e coletivo."  \cite{jogoPratPedagoc}. O que significa que contribui para o aluno se tornar um ser ativo e pensante, capacitando-o a exercer seu papel como cidadão.

"Os jogos despertam o interesse dos jovens trazendo diversos benefícios aliados à
educação[...]" \cite{appcalculo}

\begin{comment}
Gamificação foca em elementos como desafios, níveis, avatar, conquistas, histórias, pontos (Gustavo Fortes Tondello, PhD). Esses elementos são utilizados para engajamento do jogador.

Completar missões e derrotar um chefão faz o jogador se sentir competente (Gustavo Fortes Tondello, PhD).
Ser capaz de escolher diferentes caminhos ou criar coisas diferentes faz o jogador se sentir autônomo (Gustavo Fortes Tondello, PhD).
\end{comment}

Tem também o estudo \cite{appcalculo} que é uma proposta de aplicativo gamificado para ensino de cálculo onde é proposto um jogo para o ensino de matemática, porém o conteúdo abordado é de conjunto, limite, derivada e integral, não é voltado para o tema de EDO 1ª ordem.

No estudo \cite{revbibmatgam} foi feito uma revisão bibliográfica nas bases de dados Scielo, Science Direct, ACM Library, IEE Xplore Digital entre outras, para levantar o uso de gamificação e dificuldades matemáticas. De 2008 trabalhos, nenhum eram relacionando gamificação e dificuldades de matemática. No entando para \cite{dicheva} que é citado no estudo, a falta de pesquisa na área é justificada por ser uma temática nova.

As referências citadas neste tópico reforçam que jogos podem ser utilizados para fazer os estudantes gostarem e se atreverem mais no contexto da matemática. Espera-se que os alunos busquem e tenham a vontade do conhecimento por si próprio para que se tornem mais independentes. Também conclui-se que existem poucos estudos na área de C2, específico para ED, apesar de terem estudos na área de matemática, estes destinam-se a cálculo 1 e matérias do ensino fundamental.

\section[Contribuição de jogos para ensino inclusivo]{Contribuição de jogos para ensino inclusivo}

O trecho a seguir mostra o lúdico como um facilitador para inserir alunos com déficits em uma sociedade. 

"[...] a percepção do lúdico passa a ser vista como uma aliada para os professores no que tange a orientar os alunos portadores de necessidades especiais em busca do desenvolvimento das suas habilidades e potencialidades dentro de uma perspectiva inclusiva"  \cite{jogoPratPedagoc}.

De acordo com \cite{jogoPratPedagoc} o jogo é um benefício no ambiente inclusivo pois ajuda a adquirir conhecimento.

É pretendido utilizar a teconologia como ferramenta de aprendizado lúdico. O jogo é importante e eficaz para escolas inclusivas \cite{jogoPratPedagoc}.

\begin{comment}
Foram anotadas algumas observações sobre os experimentos realizados em estudantes com TDAH para ser lembrado de incluir nos requisitos do jogo para aumentar o público alvo. A seguir alguns itens: 

Crianças com TDAH precisam de um feedback mais frequente. O feedback deve ser apresentado no momento do comportamento visado. Porém o próprio feedback pode causar uma distração, então ele deve ser feito seguido de uma chamada de atenção que redirecione o aluno para a próxima tarefa a ser cumprida (George J. DuPaul, PhD e Gary Stoner, PhD. p. 131-132).

O nível de complexidade das equações e das dificuldades aumentará gradativamente. Para pessoas com TDAH a facilidade em jogar é maior quando inicia-se com instruções iniciais simples e com pouco número de etapas. Para fixar melhor o entendimento do aluno sobre a atividade pede-se que a pessoa repita o que foi pedido como objetivo para completar a atividade. Ao existirem erros em atividade é bom variar as atividades para evitar o cansaço por repetição (George J. DuPaul, PhD e Gary Stoner, PhD. p.132).

Recompensas devem ser dadas para que a motivação aumente. Porém a pessoa deve escolher o prêmio desejado ao invés de ser entregue um prêmio que ela não deseje. Para isso antes da atividade é necessário a fase de negociação para que seja escolhidos possíveis prêmios que sirvam como motivação (George J. DuPaul, PhD e Gary Stoner, PhD. p.133).

\end{comment}


\section[Suporte da tecnologia com jogos nos colégios]{Suporte da tecnologia com jogos nos colégios}
Queremos fortalecer que gamificação também pode ser combinada para ajudar a incluir os alunos com dificuldades especiais na faculdades e no meio social. "[...] a tecnologia é uma grande aliada para a aprendizagem dos alunos portadores do TDAH."  \cite{matEtdah1}.

O computador e tecnologias como celular, além de serem ferramentas de auxílio, são também  motivadoras para os estudantes \cite{softwaregamificado}.

É reforçado em \cite{revbibmatgam} que a tecnologia deve ser usada como ferramenta auxiliadora para o ensino de matemática em colégios inclusivos.

Este tópico visa reafirmar o potencial dos jogos tecnológicos para ser usado como ferramenta de atração dos alunos para os colégios e universidades.