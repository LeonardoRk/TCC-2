\chapter[Metodologia]{Metodologia}
A primeira parte da metodologia seguida foi a pesquisa bibliográfica e a segunda parte do trabalho foi um estudo de caso. A pesquisa bibliógráfica visa pesquisar e conhecer a literatura existente para levantar um referencial teórico a respeito do tema a ser tratado. Foi utilizada para levantar o referencial teórico a respeito do auxílio de mapas conceituais, a contribuição efetiva de jogos e seu sucesso no ensino.
Como referência para escolher artigos foi utilizado o artigo de \cite{howtoread} que explica como ler um artigo para saber se vamos utilizar um estudo, ou abandonar o estudo no primeiro passo, que já é possível identificar se o estudo será ou não relevante para a pesquisa. São 3 passos progressivos onde ao final de cada passo é possível decidir se o estudo está ou não relacionado com o seu tema de escrita. Os passos são:

\begin{enumerate}
\item{Passo 1:} Dar uma lida atenciosa no título do trabalho, no resumo, na introdução, na conclusão e nas referências utilizadas no trabalho. Passar o olho por cada seção para conhecer a estrutura ignorando o conteúdo. Nas referências, deve-se prestar atenção nos estudos que já foram vistos anteriormente à leitura. Nesta etapa deve-se utilizar aproximadamente 15 minutos, a depender do tamanho do artigo.

\item{Passo 2:} Ler o artigo inteiro porém não dando muita atenção para as 'provas' no artigo. Se atentar às imagens, tabelas e gráficos, observando se os eixos dos gráficos foram bem escolhidos e estão bem explicados. Este passo faz compreender o conteúdo do artigo mas não em maiores detalhes. Nesta etapa deve-se anotar as referências para futuras leituras. Para leitores inexperientes demora aproximadamente 1 hora o passo 2, porém varia dependendo do tamanho do artigo.

\item{Passo 3:} Repetir o experimento lido se atentando bem a cada detalhe, percebendo erros que o escritor do artigo pode ter experienciado e detalhes ou proposições que o autor assumiu no experimento. Este passo te faz compreender toda a profundidade do artigo. Lembre-se de anotar ideias para trabalhos futuros. O tempo utilizado é aproximadamente 4 a 5 horas para iniciantes e aproximadamente 1 hora para pesquisadores experientes. No fim deste passo deve-se ser possível replicar toda a estrutura do trabalho de memória.
\end{enumerate}

Estes passos foram utilizados para a escolha dos estudos do referencial teórico.
A segunda parte da metodologia foi o estudo de caso aplicado em alunos de C2 na universidade do Gama, utilizando o mapa conceitual e estatísticas levantadas do aprEnDO para avaliar o impacto do aplicativo como ferramenta de suporte ao aprendizado em EDO de 1º nível.

Serão solicitados aos alunos interessados para realizar um mapa conceitual a respeito do domínio do conteúdo de equações diferenciais. Espera-se que eles desenhem em grupos de 3 ou 4 pessoas os seus conhecimentos até o momento sobre equações diferenciais. Após a 1ª leva de desenhos será ministrado o conteúdo da matéria para os alunos que durará cerca de 1 mês. Um grupo de alunos sorteados usará o aplicativo e outra parte da turma não jogará. Durante os dias que estará sendo ensinado a matéria o grupo escolhido estará treinando com o jogo também e enviando suas estatísticas do jogo para um servidor central hospedado no heroku. Após o ensino do conteúdo será solicitado novamente aos grupos para desenharem um mapa conceitual sobre o mesmo conteúdo. Para analisar a evolução dos mapas conceituais, será necessário comparar os mapas do grupo que utilizou o jogo com os do grupo que não utilizou.
Espera-se encontrar no mapa os conceitos de ordem de ED, tipo, linearidade, homogeneidade e técnicas de solução de uma EDO de 1ª ordem separável, exata, não exata, homogênea e não homogênea. Estes são os temas abordados no jogo e exigido de conhecimento para poder avançar na fases dos módulos.

A análise dos mapas conceituais se dará qualitativamente, pois fica subjetivo a maneira como os alunos selecionam os conceitos, as palavras de enlace e suas proposições, e também faz-se necessário avaliar se as proposições estão corretas e se as ligações criadas são verdadeiras.

A análise de estatísticas enviadas do jogo será quantitativa e exibirá um gráfico de análise. A respeito do jogo, como colher dados? Os dados são colhidos quando o jogador está utilizando o jogo. Porém o dados são mantidos apenas no dispositivo. A pessoa pode enviar a estatística quando desejar, o professor solicitará o envio constante dos dados dos alunos.

Como enviar estatísticas? Haverá um botão responsável de enviar estatísticas, que podem ser enviadas para o servidor através do jogo em apenas um clique e uma confirmação do jogador. Assim que enviadas, as estatísticas armazenadas no celular se apagam para que no próximo envio não hajam informações repetidas.

Que dados estão sendo colhidos e enviados? As informações levantadas do jogo são separadas por pergunta, fase e módulos.

Por pergunta:
	- Qual é a pergunta alvo
	- Número das 4 EDs
	- Quantas tentativas foram necessárias para acertar a pergunta. **para saber se clicam aleatório ou escolhem antes de clicar **
	- Tempo gasto em cada pergunta  **para saber se a tela não ficou parada e o celular sem atenção**

	Algumas observações são: o clique pode demorar para acontecer ou acontecer muito rápido e as consequências podem ser acertar de primeira ou não acertar de primeira.
	Algumas inferências são: 
	cliques muito rápidos e não acertar de primeira == PODE significar chute aleatório
	cliques muito rápidos e acertar de primeira == pode ser um robô? alunos muito bem preparados?
	cliques devagar e não acertar de primeira == tá muito difícil? O celular está parado? 
	cliques devagar e acertar de primeira == estava pensando? Estava resolvendo? O celular estava parado?

Por fase:
	- Quanto tempo durou cada fase (de 20 perguntas)

Dentro de cada módulo:
	- Quantas vezes clickou para jogar em uma fase(linear,exata,etc..) de classificação e saiu? (após 3 segundos)
	- Quantas vezes clickou para jogar em uma fase(homog, exata, etc) de resolução e saiu? (após 3 segundos)
	- Tempo total no módulo de classificação
	- Tempo total no módulo de resolução
	- Quantas vezes entrou e saiu de cada módulo (considerar que saiu é após 3 segundos)


Quantas vezes cada módulo foi clickado para jogar, e quantas ficaram sendo jogados mesmo.

Será avaliado dos mapas conceituais se houve evolução significativa entre os mapas conceituais dos alunos participantes em relação aos não participantes do jogo.
As estatísticas do jogo será para avaliar o desempenho dos alunos e qual a aceitação do jogo pelos participantes, para saber se é uma estratégia que pode ser levada adiante ou não.