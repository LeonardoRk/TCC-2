\chapter[Metodologia]{Metodologia}
A metodologia a ser seguida será a pesquisa bibliográfica. Este tipo de metodologia visa pesquisar e conhecer a literatura existente para levantar um referencial teórico a respeito do tema a ser tratado. Serão utilizados a pesquisa de artigos nas bases de dados da CAPES com as credenciais de aluno da UnB para acessar os estudos. Não serão aceitos artigos pagos. 
Como referência para escolher artigos foi utilizado o artigo \cite{howtoread} para evitar gastar tempo desnecessário em artigos que podem não estar relacionados ao tema da pesquisa.
Com o referencial teórico consolidado para o apoio de jogos e tecnologias ao ensino de matemática, será explicado o funcionamento do jogo e como jogar cada nível. Com o jogo definido, protótipo desenhado, features e requisitos definidos, será explicado e explicitado as tecnologias a serem utilizadas e como será o método de população das equações para que seja possível existir o jogo.
Após a fase de planejamento aprovada inicia-se o desenvolvimento do jogo com entregas ágeis e contínuas no decorrer das \textit{sprints}.
Tendo o jogo pronto para uso, será aplicado em uma turma de cálculo 2 no 1º semestre de 2019 para fazer um comparativo com outra turma de C2 e avaliar se houve alguma melhoria de desempenho. As medições serão feitas através de um questionário que ainda não foi decidido se será qualitativo ou quantitativo e também através dos dados estatísticos compartilhados pelos jogadores do jogo.




\begin{comment}
 Podem ser seguidos até 3 passos para utilizar um estudo, ou abandonar o estudo no primeiro passo que já é possível identificar se o estudo será ou não relevante para a pesquisa. 
}
\begin{itemize}
\item{Passo 1:} Dar uma lida atenciosa no título do trabalho, no resumo, na introdução, na conclusão e nas referências utilizadas no trabalho. Passar o olho por cada seção ignorando o conteúdo. Nas referências, deve-se prestar atenção nos estudos que já foram vistos. Nesta etapa deve-se utilizar aproximadamente 15 minutos, a depender do tamanho do artigo.

\item{Passo 2:} Ler o artigo inteiro porém não dando muita atenção para as 'provas' no artigo. Se atentar às imagens, tabelas e gráficos, observando se os eixos dos gráficos foram bem escolhidos e estão bem explicados. Este passo faz compreender o conteúdo do artigo mas não em detalhes. Nesta etapa deve-se anotar as referências para futuras leituras. Para leitores inexperientes demora aproximadamente 1 hora o passo 2, varia dependendo do tamanho do artigo.

\item{Passo 3:} Repetir o experimento se atentando bem a cada detalhe, percebendo erros que o escritor do artigo pode ter experienciado, e detalhes que o autor assumiu no experimento. Este passo te faz compreender toda a profundidade do artigo. Lembre-se de anotar ideias para trabalhos futuros. O tempo utilizado é aproximadamente 4 a 5 horas para iniciantes e aproximadamente 1 hora para pesquisadores experientes. No fim deste passo deve-se ser possível replicar toda a estrutura do trabalho de memória.
\end{itemize}
\end{comment}

