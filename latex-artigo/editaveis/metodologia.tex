\chapter[Metodologia]{Metodologia}
A metodologia utilizada foi a de relato de experiência e é de caráter quantitativo e qualitativo. Diz-se qualitativa por se preocupar com as percepções das pessoas em relação ao uso do jogo AprEnDO. Diz-se quantitativa por analisar os dados enviados dos alunos para o servidor de coleta. 

Foi levantado um referencial teórico a respeito da contribuição e o sucesso de jogos no ensino, conceitos da Engenharia de Software que podem ser levados em conta de modo a aulixiar o desenvolvimento de software e a utilização de mapas conceituais como ferramenta para verificar a aderência de conhecimento nos alunos.

O relato de experiência é sobre a aplicação do jogo aprEnDO em uma turma de Engenharias da UnB-FGA na disciplina de C2. Por questões éticas e em respeito aos discentes voluntários que participaram, os nomes dos alunos serão preservados. Os levantamentos ocorreram no primeiro semestre de 2019. O conteúdo da matéria ministrada aos alunos durou cerca de 1 mês. Após o período de ensino da matéria, a turma utilizou a aplicação em torno de dois meses para o reforço do conteúdo e neste tempo foram enviando suas estatísticas para um servidor central. Os dados foram consolidados em tabelas para análise e também foram analisados os mapas conceituais produzidos pelos alunos. Foram relatados \textit{bugs}, erros e falhas encontrados pelos alunos durante a fase de jogo. 