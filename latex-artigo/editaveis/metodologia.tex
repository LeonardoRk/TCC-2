\chapter[Metodologia]{Metodologia}
Utilizou-se a pesquisa bibliográfica e um relato de experiência. A pesquisa bibliógráfica visa pesquisar e conhecer a literatura existente para levantar um referencial teórico a respeito do tema a ser tratado. Esta serviu para levantar o referencial teórico a respeito do auxílio de mapas conceituais, a contribuição efetiva de jogos e seu sucesso no ensino e os conceitos da Engenharia de Software.

A segunda parte da metodologia foi um relato de experiência da aplicação do jogo aprEnDO em alunos de C2 na universidade do Gama, utilizando o mapa conceitual e estatísticas levantadas do jogo para avaliar o impacto do aplicativo como ferramenta de suporte ao aprendizado em EDO de 1º nível.

O conteúdo da matéria ministrado aos alunos durará cerca de 1 mês. Durante os dias que estará sendo ensinado a matéria, a turma estará treinando a fixação do conteúdo também com o jogo e enviando suas estatísticas para um servidor central. Após isso solicitará aos alunos interessados para realizar um mapa conceitual a respeito do domínio do conteúdo de equações diferenciais. Espera-se que eles produzam em grupos de 3 ou 4 pessoas os seus conhecimentos até o momento sobre equações diferenciais. 

Espera-se encontrar no mapa os conceitos de ordem de ED, tipo, linearidade, homogeneidade e técnicas de solução de uma EDO de 1ª ordem separável, exata, não exata, homogênea e não homogênea. Estes são os temas abordados no jogo e exigido de conhecimento para poder avançar na fases dos módulos.

A análise dos mapas conceituais se dará qualitativamente, pois fica subjetivo a maneira como os alunos selecionam os conceitos, as palavras de enlace e suas proposições, e também faz-se necessário avaliar se as proposições estão corretas e se as ligações criadas são verdadeiras.

A análise de estatísticas enviadas do jogo será quantitativa e exibirá um gráfico de análise. 

Todos os alunos da Faculdade do Gama que cursam C2 no período 1/2019 independente da turma, participarão de uma prova compartilhada formulada pelos professores. Essa prova visa medir o conhecimento dos alunos de diferentes turmas. Uma das questões será de EDO 1ª ordem, esta questão será utilizada para a comparação entre os alunos participantes do jogo e o restante não participante. A questão será um indicador de melhoria utilizado para a análise.

