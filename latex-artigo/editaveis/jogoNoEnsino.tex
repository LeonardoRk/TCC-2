
\section[Sucesso de jogos no ensino]{Sucesso de jogos no ensino}

Alguns autores diferenciam jogos e gamificação, porém neste estudo ambos serão tratados de maneira igual querendo se referir à atividades lúdicas. Este tópico visa apoiar jogos e gamificação como uma estratégia boa para aprendizado dos jogadores e também como boa ferramenta para ser utilizada como suporte à aprendizagem.  

Acredito que para tornar mais eficiente a absorção do conhecimento dos alunos, para analisar o ensino/aprendizado devemos olhar da esfera do alunos e não dos professores. Como assim? Ao invés do professor ensinar para o aluno, visão essa que torna o aluno como reativo, devemos olhar do ponto em que o aluno aprende com o professor. Desta segunda maneira é uma visão pró-ativa, onde o aluno não tem apenas o professor como única fonte de absorção de conhecimento, mas como uma das fontes. Nesta segunda visão o estudante deve ir atrás de como adquirir o conhecimento e seguir conselhos de pessoas mais experientes (como os professores). 

Um dos principais objetivos de uma instituição de ensino é além de ensinar aos alunos, garantir que estes aprendam o que foi ensinado. Há quem diga que a finalidade na sala de aula é estimular ideias dos alunos e então ensinar apenas com aulas expositivas tradicionais pode dificultar o aprendizado \cite[p. 4]{sucessoJogoEngSoft}.

Segundo \apud[p. 2257]{silva}{sucessoJogoEngSoft}, uma abordagem alternativa às aulas tradicionais, devido  a elas serem mais teóricas e expositivas, é a utilização dos jogos, pois esta abordagem preza por uma teoria de motivação humana como ponto de partida. Na literatura, existem muitos trabalhos de profissionais da educação utilizando jogos como uma ferramenta de auxílio à aprendizagem \cite[p. 3]{sucessoJogoEngSoft}. Alguns dos motivos pelo qual os autores optam pela utilização de jogos são:

\begin{itemize}
	\item Despertar o interesse dos jovens e trazer diversos benefícios à educação \cite{appcalculo}.
	\item O jogo é uma prática que ajuda na concretização do conhecimento, além de tornar o ambiente mais prazeroso \cite{jogoPratPedagoc}. 
	\item Motivação e aprendizado por descoberta \apud[p. 2]{fuku}{savi}.
	\item Desenvolvimento de habilidades cognitivas, experiência de novas identidades, socialização, coordenação motora e comportamento expert \cite[p. 3 e 4]{savi}.
	\item Poder proporcionar a vivência em experiências de aprendizagem concretas \apud[p. 4]{monsalve}{sucessoJogoEngSoft}.
\end{itemize}

Então com os itens elencados acima, acredita-se que os jogos deixam um ambiente mais dinâmico e prazeroso, por isso que eles tem um bom potencial para serem utilizados na hora de ensinar os alunos. De acordo com \cite{nunesParreira}, sabe-se que os jogos educacionais têm sido intensamente utilizados por profissionais da área de educação como auxílio para a construção do conhecimento.

Outros exemplos de jogos sendo utilizados em contextos educacionais: Este estudo \cite{sucessoJogoEngSoft} fez um levantamento de jogos para o uso específico na Engenharia de Software. São listados 20 jogos no apêndice, com cada um focado em uma disciplina do curso. 

Com todas as constatações acima, espera-se ter convencido que muitos autores já estão utilizando jogos ou aplicações como uma ferramenta de auxílio para o ensino e que em alguns casos elas são um sucesso, fazendo com que de fato os jogadores aprendam o conteúdo que o jogo deseja transmitir. Para isso este trabalho trata-se do desenvolvimento de um jogo para o ensino de matemática à alunos do ensino superior.

O computador e tecnologias como celular, além de serem ferramentas de auxílio, são também  motivadoras para os estudantes \cite{softwaregamificado}.
Uma das maneiras de ajudar os alunos a se interessarem mais em sala de aula e atrair a atenção dos mesmos é utilizar o lúdico, ou seja, aprender brincando. Para isso o uso de computadores ou tecnologias da informação como o celular é útil para melhorar o engajamento nas tarefas, principalmente com exercícios e aplicações para a prática das matérias ensinadas em sala de aula \cite{tdahNasEscolas2}.

"Uma das propostas de melhoria de aprendizado em sala de aula são os jogos educacionais" \cite[p. 4]{sucessoJogoEngSoft}


"O caráter lúdico, bem como a possibilidade de atuação crítica, proporciona ao aluno uma participação efetiva no processo de ensino aprendizagem, se tornando um momento ímpar de crescimento pessoal e coletivo." \cite{jogoPratPedagoc}. O que significa que contribui para o aluno se tornar um ser ativo e pensante, capacitando-o a exercer seu papel como cidadão.


"Uma das propostas metodológicas para ensino de engenharia de software e suas disciplinas, são os jogos educacionais. 


\begin{citacao}
Foram pesquisados nos Anais da base WEI e selecionados os artigos que tivessem trabalhos com jogos que auxiliassem no aprendizado de ensino superior das disciplinas de Engenharia de Software, referencialmente os que validassem com alunos. \cite{sucessoJogoEngSoft}
\end{citacao}


Atratividade de jogos está relacionada a mecanismos psicológicos e sociais \cite{sucessoJogoEngSoft}

Para o jogo ser bem aceito e cumprir com a sua meta, ele deve dar uma boa base de conhecimento e motivação \cite{sucessoJogoEngSoft}.

Também conclui-se que existem poucos estudos de jogos na área de C2, específico para ED, apesar de terem estudos na área de matemática, estes destinam-se a cálculo 1 e matérias do ensino fundamental.


\begin{comment}
Gamificação foca em elementos como desafios, níveis, avatar, conquistas, histórias, pontos (Gustavo Fortes Tondello, PhD). Esses elementos são utilizados para engajamento do jogador.

Completar missões e derrotar um chefão faz o jogador se sentir competente (Gustavo Fortes Tondello, PhD).
Ser capaz de escolher diferentes caminhos ou criar coisas diferentes faz o jogador se sentir autônomo (Gustavo Fortes Tondello, PhD).
\end{comment}
