
\section[Sucesso de jogos no ensino]{Sucesso de jogos no ensino}

Este tópico visa apoiar jogos e gamificação como uma estratégia boa para aprendizado dos jogadores e também como boa ferramenta para ser utilizada como aprendizagem. Para isso serão mostrados citações de autores que reforçam isso que foi dito, assim também como exemplos de casos reais da utilização de jogos em ambientes educacionais. 

Jogo é prática que ajuda na concretização do conhecimento, além de tornar o ambiente mais prazeroso \cite{jogoPratPedagoc}. 

"O caráter lúdico, bem como a possibilidade de atuação crítica, proporciona ao aluno uma participação efetiva no processo de ensino aprendizagem, se tornando um momento ímpar de crescimento pessoal e coletivo." \cite{jogoPratPedagoc}. O que significa que contribui para o aluno se tornar um ser ativo e pensante, capacitando-o a exercer seu papel como cidadão.

"Os jogos despertam o interesse dos jovens trazendo diversos benefícios aliados à educação[...]" \cite{appcalculo}

"Na literatura, encontram-se vários trabalhos que demonstram profissionais de educação utilizando os jogos como ferramenta de auxílio ao aprendizado" \cite[p. 3]{sucessoJogoEngSoft}

"Como a finalidade na sala de aula é estimular ideias dos alunos, ensinar apenas com aulas expositivas
tradicionais pode dificultar o aprendizado." \cite[p. 4]{sucessoJogoEngSoft} 

"Uma das propostas metodológicas para ensino de engenharia de software e suas disciplinas, são os jogos educacionais. Sabe-se que os jogos educacionais, segundo Nunes e Parreira (2015), têm sido intensamente utilizados por profissionais da área de educação como auxílio para a construção do conhecimento. Em sua pesquisa, Fukusawa et al. (2015) apontam alguns dos benefícios que os jogos educacionais podem trazer ao
processo de ensino e aprendizagem como, por exemplo, a motivação e o aprendizado por descoberta. Portanto, os jogos podem proporcionar a vivência em experiências de aprendizagem concretas \apud[p. 4]{monsalve}{sucessoJogoEngSoft}."

Silva et al. (2015) comentam que uma abordagem alternativa às aulas tradicionais, devido  a elas serem mais teóricas e expositivas, é a utilização dos jogos, pois esta abordagem preza por uma teoria de motivação humana como ponto de partida \cite{sucessoJogoEngSoft}.

"Uma das propostas de melhoria de aprendizado em sala de aula são os jogos educacionais" \cite[p. 4]{sucessoJogoEngSoft}

\begin{citacao}
Foram pesquisados nos Anais da base WEI e selecionados os artigos que tivessem trabalhos com jogos que auxiliassem no aprendizado de ensino superior das disciplinas de Engenharia de Software, referencialmente os que validassem com alunos. \cite{sucessoJogoEngSoft}
\end{citacao}

Jogos e gamificação diferem-se, porém ambos já vem sendo usados em ambientes de ensino. 
Existem vários exemplos de casos de sucesso, como por exemplo "O bicho papão da matemática virou um gatinho", é um jogo de matemática para alunos do 1º e 2º ano fundamental.O jogo ajuda a fazer divisões. Segundo a notícia no link \url{http://portal.mec.gov.br/component/content/article?id=72701}, 300 alunos se beneficiaram deste projeto. Um dos símbolos que ficou marcado era de uma aluna que tinha reprovado, tirava notas baixas, não interagia muito com os outros alunos e acabou se envolvendo, aprendendo o jogo de tal maneira que passou a tirar 10, ir resolver no quadro e se sentir capaz. É relatado também que não melhorou só em matemática, como em outras matérias. Segundo o professor, além da menina citada, muitos outros que não sabiam divisão aprenderam também.

Outros exemplos de jogos sendo utilizados em contextos educacionais: Este estudo \cite{sucessoJogoEngSoft} fez um levantamento de jogos para o uso específico na engenharia de software. São listados 20 jogos no apêndice, com cada um focado em uma disciplina do curso. 

Mais um estudo \cite{appcalculo}, que é uma proposta de aplicativo gamificado para ensino de cálculo onde é proposto um jogo para o ensino de matemática com o conteúdo voltado para os temas de  conjunto, limite, derivada e integral, não é voltado para o tema de EDO 1ª ordem.

Então até aqui é possível ver que existem muitos jogos no ensino. Porém jogos para matemática no ensino superior não foram encontrados muitos trabalhos, e afunilando um pouco mais para jogos de equação diferencial não foi encontrado nenhum.

Atratividade de jogos está relacionada a mecanismos psicológicos e sociais \cite{sucessoJogoEngSoft}

Para o jogo ser bem aceito e cumprir com a sua meta, ele deve dar uma boa base de conhecimento e motivação \cite{sucessoJogoEngSoft}.

O computador e tecnologias como celular, além de serem ferramentas de auxílio, são também  motivadoras para os estudantes \cite{softwaregamificado}.

Este tópico visa reafirmar o potencial dos jogos tecnológicos para ser usado como ferramenta de atração dos alunos para os colégios e universidades e as citações neste tópico visam reforçar que jogos podem ser utilizados para fazer os estudantes gostarem e se atreverem mais no contexto da matemática. Espera-se que os alunos busquem e tenham a vontade do conhecimento por si próprio para que se tornem mais independentes. Também conclui-se que existem poucos estudos de jogos na área de C2, específico para ED, apesar de terem estudos na área de matemática, estes destinam-se a cálculo 1 e matérias do ensino fundamental.


\begin{comment}
Gamificação foca em elementos como desafios, níveis, avatar, conquistas, histórias, pontos (Gustavo Fortes Tondello, PhD). Esses elementos são utilizados para engajamento do jogador.

Completar missões e derrotar um chefão faz o jogador se sentir competente (Gustavo Fortes Tondello, PhD).
Ser capaz de escolher diferentes caminhos ou criar coisas diferentes faz o jogador se sentir autônomo (Gustavo Fortes Tondello, PhD).
\end{comment}
