
\section[Sucesso de jogos no ensino]{Sucesso de jogos no ensino}

Alguns autores diferenciam jogos e gamificação, porém neste estudo ambos serão tratados de maneira igual querendo se referir à atividades lúdicas. Esta seção visa apoiar jogos como uma estratégia boa para aprendizado dos alunos/jogadores e também como uma boa ferramenta para ser utilizada como aprendizagem. Espera-se que com o uso de jogos os estudantes gostem e se atrevam mais no contexto da Matemática buscando por conta própria o conhecimento e que se tornem mais independentes.

No século XXI com todo o avanço tecnológico o celular se tornou um aparelho indispensável para o ser humano, então nada melhor do que unir o útil ao agradável e utilizá-lo como uma ferramenta de auxílio e de motivação para os estudantes \cite{softwaregamificado}. O uso de computadores ou tecnologias da informação como o celular é útil para melhorar o engajamento nas tarefas, principalmente com exercícios e aplicações para a prática das matérias ensinadas em sala de aula \cite{tdahNasEscolas2}.

Como a finalidade na sala de aula é estimular ideias dos alunos, ensinar apenas com aulas expositivas tradicionais pode dificultar o aprendizado. Por isso os jogos chegaram às salas de aula para ajudar \cite[p. 4]{sucessoJogoEngSoft}. De acordo com o dicionário Michaelis, uma das definições de jogo é "Qualquer atividade recreativa que tem por finalidade entreter, divertir ou distrair" \cite{Michaelis}. Alguns sinônimos são: brincadeira, entretenimento, diversão, entre outros. O jogo tem além do seu papel lúdico, a característica da construção de conhecimento, ou seja, fazer com que o jogador aprenda através das experiências adquiridas por ele \cite{jogoPratPedagoc} \cite{appcalculo} \cite{Nunes} \cite{fukusawa}. De acordo com \cite{jogoPratPedagoc}, o caráter lúdico proporciona ao aluno uma participação efetiva no processo de ensino aprendizagem se tornando um momento ímpar de crescimento pessoal e coletivo. 

Os jogos e a gamificação se mostraram como uma estratégia efetiva tanto no ensino fundamental quanto no ensino superior. Na literatura, encontram-se vários trabalhos que demonstram profissionais de educação utilizando os jogos como ferramenta de auxílio ao aprendizado \cite[p. 3]{sucessoJogoEngSoft}. Abaixo seguem alguns exemplos.

\begin{itemize}
  \item Um jogo chamado "O bicho papão da Matemática virou um gatinho" que ensina Matemática para alunos do 1º e 2º ano fundamental. Este ajuda os alunos a fazer divisões. Segundo a notícia disponível no \href{http://portal.mec.gov.br/component/content/article?id=72701}{portal mec}, 300 alunos se beneficiaram deste projeto. Um dos símbolos que ficou marcado era de uma aluna que tinha reprovado, tirava notas baixas, não interagia muito com os outros alunos e acabou se envolvendo, aprendendo o jogo de tal maneira que passou a tirar 10, ir resolver no quadro e se sentir capaz. É relatado também que não melhorou só em Matemática, como em outras matérias. Segundo o professor, além da menina citada, muitos outros alunos que não sabiam divisão aprenderam também.


  \item O estudo \cite{sucessoJogoEngSoft} realizou uma pesquisa nos Anais da base WEI para selecionar artigos que contém jogos que auxiliam no aprendizado das disciplinas de Engenharia de Software. Foram selecionados preferencialmente os estudos que validassem os resultados com alunos \cite{sucessoJogoEngSoft}. Como resultado foi obtido um levantamento de 20 jogos focados em disciplinas do curso, dentre elas gestão de projetos, algoritmos, estrutura de dados, teste de software entre outras.

  \item O estudo \cite{appcalculo} é uma proposta de aplicativo gamificado para ensino de Cálculo 1, onde é proposto um jogo para o ensino de Matemática com o conteúdo voltado para os temas de conjunto, limite, derivada e integral.

  \item O estudo de \cite{jogoSuporteMat} cita um jogo educacional de matemática com um agente que ensina aritmética que é capaz de aprender. O programa utiliza de inteligência artificial.   
\end{itemize}


Alguns dos motivos pelo qual os profissionais da educação optam pela utilização de jogos são:

\begin{itemize}
	\item Despertar o interesse dos jovens e trazer diversos benefícios à educação \cite{appcalculo}.
	\item O jogo é uma prática que ajuda na concretização do conhecimento, além de tornar o ambiente mais prazeroso \cite{jogoPratPedagoc}. 
	\item Motivação e aprendizado por descoberta \apud[p. 2]{fukusawa}{savi}.
	\item Desenvolvimento de habilidades cognitivas, experiência de novas identidades, socialização, coordenação motora e comportamento expert \cite[p. 3 e 4]{savi}.
	\item Poder proporcionar a vivência em experiências de aprendizagem concretas \apud[p. 4]{monsalve}{sucessoJogoEngSoft}.
\end{itemize}

De acordo com \cite{Nunes}, sabe-se que os jogos educacionais têm sido intensamente utilizados por profissionais da área de educação como auxílio para a construção do conhecimento e para motivar os alunos, porém os resultados encontrados mostraram que apesar de existirem muitos jogos, poucos são de Matemática. Dos jogos de Matemática encontrados, foi constatado que uma minoria deles são para o ensino superior, e dos jogos de Matemática para o ensino superior apenas um deles falou de equações diferenciais, que é o estudo \cite{videoGameED} o qual aplica os conceitos na movimentação dos personagens (princípio da dinâmica de Newton), porém o jogo não é focado no ensino de equação diferencial ordinária.

O estudo de \cite{revbibmatgam} é uma revisão sistemática de literatura realizada nas bases de dados Scielo Library, BIREME Biblioteca, Science Direct, ACM Library e IEEE Xplore Digital Library e nos periódicos Revista Brasileira de Informática na Educação e a Revista de Novas Tecnologias na Educação. O estudo procurou artigos que constatavam a existência de ferramentas relacionadas com gamificação e dificuldades de aprendizagem de matemática. De 2008 trabalhos, nenhum eram relacionando gamificação e dificuldades de matemática. Por fim o estudo concluiu que:

\begin{citacao}
identifica-se a necessidade de pesquisas sobre esta temática, já que as dificuldades de aprendizagem na Matemática são frequentes em sala de aula, e a gamificação tem-se mostrado uma ferramenta promissora nos ambientes de ensino e aprendizagem em todos os níveis de ensino. 
\end{citacao}

No entanto para \cite{dicheva} que é citado no estudo \cite{revbibmatgam}, a falta de pesquisa nesta área é justificada por ser uma temática nova. Acredita-se que com o tempo o número de interessados e pesquisas na área aumentará.

Com todas as constatações acima, espera-se ter convencido que muitos autores já estão utilizando jogos ou aplicações como uma ferramenta de auxílio para o ensino e que em alguns casos elas são um sucesso, fazendo com que de fato os jogadores aprendam o conteúdo que o jogo deseja transmitir. Constatando que há poucos jogos de Matemática para equações diferenciais este trabalho trata-se do desenvolvimento de um jogo para o ensino de matemática à alunos do ensino superior.
