\section[Equação diferencial (ED)]{Equação diferencial (ED)}
 


\subsubsection{Solução de EDO}



\subsection{EDO exata}
Sendo uma solução exata é possível resolver da seguinte maneira: \\

Como resolução da exata é preciso integrar um lado considerando uma variável constante e chamálo de W, é provado que o outro lado menos derivada parcial de W em relação á outra variável diferente da integrada anteriormente é uma função apenas dessa variável usada para integrar.

	Então a solução  é integrar um lado + a integral do outro lado menos derivada parcial da primeira parte relacionado á outra variável é igual a uma constante K, que é a solução geral, caso haja algum PVI pode ser substituído para encontrar a solução particular. 

\end{itemize}
































\begin{itemize}
\item{ED homogênea:} São as equações que pode-se fazer uma mudança de variável e colocar a nova variável em evidência em relação a equação inteira. Por exemplo em uma equação f(x,y) substitui-se por f(kx,ky) e é possível escrever na forma k * f(x,y).
\end{itemize}

Exemplo:

$ f(x,y) = x^2 – 3xy+5y^2  $

\begin{equation}
f(kx,ky) =(kx)^2 – 3(kx)(ky) +5(ky)^2 = k^2 x^2 –3k^2 xy+5k^2 y^2 
\end{equation}
\begin{equation}
f(kx,ky) = k 2 [ x 2 –3xy+5y 2 ] = k 2 f(x,y) 
\end{equation}


\begin{itemize}
\item{ED exata:} 
\end{itemize}

\begin{equation}
M(x,y)x + N(x,y) = 0
\end{equation}
e precisa satisfazer a igualdade de

\begin{center}
$ \frac{\partial M}{\partial y} = \frac{\partial N}{\partial x} $
\end{center}



\begin{itemize}
\item{EDO linear 1ª ordem e não linear:} São equações ordinárias que tem a variável dependente e todas suas derivadas elevadas à primeira potência.
Método de resolução: 

\begin{comment}
  Exemplo de uma EDO linear 1ª ordem 

  a1(x)\frac{dy}{dx} + a0(x)y = g(x)
\end{comment}
\end{itemize}


