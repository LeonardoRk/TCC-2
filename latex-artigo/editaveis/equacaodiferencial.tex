\chapter[Equação diferencial]{Equação Diferencial}
Na engenharia e na natureza existem problemas e fenômenos que envolvem tempo, distância, tamanho, velocidade, volume entre outros. É possível fazer modelagens desses casos e relacioná-los a equações. Em alguns casos essas equações incógnitas envolvem uma taxa de variação, quando isso ocorre dizemos que as equações estão relacionadas às chamadas equações diferenciais (ED).

Equações diferenciais envolvem derivadas de uma ou mais variáveis dependentes em relação a uma ou mais variáveis independentes \cite{explicacaoEDO}.

\section[Classificação de ED]{Classificação de ED}

As ED podem ser classificadas por \textbf{tipo}, \textbf{ordem} e \textbf{linearidade}.

\subsection[Tipo]{Tipo}

Podem ser ED ordinárias ou parciais, depende do número de variáveis independentes. Quando a ED tem apenas uma variável independente, é chamada de \textbf{ED ordinária (EDO)}. Quando a ED tem mais que uma variável independente, é chamada de \textbf{ED parcial (EDP)}.
Uma EDP é representada pelo símbolo $ \partial $, normalmente chamado de \textit{del}.

Nas tabelas \ref{tabelaEdo} e \ref{tabelaEDP}  é possível ver exemplos de EDO e EDP.

\begin{table}[h!]
\parbox{.40\linewidth}{
\centering
\begin{tabular}{c}
\hline 
  $
\dfrac{dy}{dx} = x^2y
$
  \\  \\
  
$
\dfrac{dy}{dx} = sen(x)
$

 \\ \\
  
$
\dfrac{d^2y}{dx^2} + x \dfrac{dy}{dx} + y = 0
$  \\

\hline
\end{tabular}
\caption{Exemplos de EDO}
\label{tabelaEdo}
}
\hfill
\parbox{.50\linewidth}{
\centering
\begin{tabular}{c}
\toprule

$
\dfrac{\partial ^2u}{\partial x^2} + \dfrac{\partial ^2u}{\partial t^2} = 0  \quad \textrm{,} \quad  u = f(x,t)
$

 \\ \\
$
\dfrac{\partial^2u}{\partial x^2} + 5 \dfrac{\partial u}{\partial t} + 3u = 0 \quad \textrm{,} \quad   u = f(x,t)
$

 \\ \\ 
 $ \dfrac{\partial u}{\partial t}(x,t) = 3 \dfrac{\partial^2u}{\partial x^2}(x,t) $  
 \\

\bottomrule
\end{tabular}
\caption{Exemplos de EDP}
\label{tabelaEDP}
}
\end{table}

Na tabela \ref{tabelaEdo} é possível observar que a variável dependente y é derivada apenas em relação à variável independente x.

Na tabela \ref{tabelaEDP} é possível observar que a variável dependente u é dependente das variáveis x e t independentes.


\subsection[Ordem]{Ordem}

Uma EDO pode ser classificada de ordem 1 até n. A ordem é um número inteiro que representa o número máximo de derivadas de uma função. Abaixo seguem exemplos de diferente ordem das EDOs. 

\begin{center}
\begin{table}[h!]
\centering
\begin{tabular}{l  p{0.4\linewidth}}

\toprule
$  y'' - (10y')^4 + 37y = 0  $ & Exemplo de EDO de segunda ordem \\
\\
$ dy/dx + sen(x) - y = 1 $ & Exemplo de EDO de primeira ordem \\ 
\\
 $
\dfrac{\partial^2u}{\partial x^2} + 5 \dfrac{\partial u}{\partial y} - 3x = 0 \quad \textrm{,} \quad   u = f(x,y)
$
 &  
 Exemplo de EDP de segunda ordem 
 \\
 
\bottomrule


\end{tabular}
\caption{Ordem de equações diferenciais}
\label{tabelaOrdem}
\end{table}
\end{center}

Na tabela \ref{tabelaOrdem} o primeiro exemplo é uma EDO de ordem 2, tendo em vista que $(10y')^4$ é uma derivada de primeira ordem elevado à quarta potência.
O segundo exemplo é uma EDO de primeira ordem, pois a maior quantidade de derivadas presente é 1.
No terceiro exemplo temos uma EDP, pois a variável dependente u está derivada 2 vezes em relação à variável independente x e derivada 1 vez em relação à independente y. Portanto como o maior número de derivadas é 2, nos dá uma derivada de segunda ordem.

\subsection[Linearidade]{Linearidade}
As equações diferenciais podem ser classificadas em linear e não-linear.
Uma equação linear é aquela que possui apenas funções lineares no lado esquerdo e direito da igualdade. A seguir um exemplo da forma geral de uma equação linear.
\begin{center}
\begin{equation} \label{formaGeral}
 \operatorname{a}_{n}(x) \dfrac{d^ny}{dx^n} + \operatorname{a}_{n-1}(x)\dfrac{d^{n-1}y}{dx^{n-1}} + ... + \operatorname{a}_{1}(x)\dfrac{dy}{dx} + \operatorname{a}_{0}(x)y = g(x) 
\end{equation}
\end{center}

Para ser linear, é necessário cumprir 2 condições. 
\begin{itemize}
	\item{}A variável dependente y e todas as suas derivadas devem ter grau 1 (elevado a 1)
	\item{}Cada coeficiente é dependente apenas de 1 variável independente x.
\end{itemize}

\begin{center}
\begin{table}[h!]
\centering
\begin{tabular}{p{1.2cm} l  p{0.4\linewidth}}

\toprule
1) & $  -x^2y''' + 3xy'' + 2y = 0  $ &  ED linear \\
\\
2) &  $ 2x\frac{d^3y}{dx^3} + (x- 4)y = 0 $  &  ED linear  \\
\\
3) & $ yy'' - (2x -3)y' - 1y  = 3x $ & ED não linear \\ 
\\
4) & $ x\frac{d^3y}{dx^3} + (\frac{d^2y}{dx^2})^2 = 0 $ & ED não linear \\ 
\\
5) & $ xdy + (y - xy - e^x)dx = 0 $ & ED linear \\ 
 
\bottomrule

\end{tabular}
\caption{ED lineares e não lineares}
\label{tabelaLinearidade}
\end{table}
\end{center}

Na tabela \ref{tabelaLinearidade} é possível ver exemplos de ED lineares e não lineares.
O primeiro e o segundo exemplo são ED lineares pois cumprem as 2 propriedades, já que o termo independente y e todas suas derivadas tem grau 1 e todos os seus coeficientes estão apenas em função da variável independente x.

O terceiro exemplo da tabela \ref{tabelaLinearidade} apesar de ter o termo independente y e todas suas derivadas de grau 1, apresenta um coeficiente em função da variável dependente y, no termo $ yy''$

O quarto exemplo da tabela \ref{tabelaLinearidade} também é não-linear pois apresenta a derivada de ordem 2 elevado ao grau 2, descumprindo com a propriedade de ter apenas termos lineares.

O quinto exemplo da tabela \ref{tabelaLinearidade}, apesar de não estar escrito na mesma forma de \ref{formaGeral} também é uma ED linear. A equação $ xdy + (y - xy - e^x)dx = 0 $ pode ser reescrita para a forma \ref{formaGeral}, quando dividida a equação por dx,
\begin{center}
$ x\frac{dy}{dx} + (y - xy - e^x) = 0 $ 
\end{center}
e colocando y em evidência e somando $e^x$ de ambos os lados:
\begin{center}
$ x \frac{dy}{dx} + (1 - x)y =  e^x $
\end{center}

Então é possível notar que a variável y dependente e todas suas derivadas tem grau 1 e os coeficientes estão função da variável independente x, caracterizando-a como linear.

Para uma ED não ser linear basta que não cumpra 1 das duas propriedades citadas.

\section[Solução de ED]{Solução de ED}

Resolver uma ED significa encontrar a função que satisfaça a equação diferencial. É necessário integrar uma diferencial para encontrar a solução.
Para dizer que uma equação soluciona uma EDO, basta que qualquer função f definida em algum intervalo I ao ser substituída na equação diferencial reduza a equação a uma identidade \cite{explicacaoEDO}.

\begin{flushleft}
Por exemplo, considere a equação diferencial abaixo e a sua solução 
\end{flushleft}

\begin{equation} \label{edExemplo} y' = 25 + y^2 \end{equation}  \begin{equation} y = 5tg(5x) \end{equation} A equação y é considerada solução, pois ao se substituir y e sua derivada na ED \ref{edExemplo} é encontrada a identidade 0 = 0.

\begin{center}

\begin{flushleft}
Como y = 5tg(5x), então \begin{equation}  y' = 5 * sec^2(5x) * 5,  \end{equation} 
\end{flushleft}

\begin{flushleft}
que é igual a \begin{equation} y' = 25sec^2(5x).  \end{equation}
\end{flushleft}

\begin{flushleft}
Substituindo y e y' na equação teremos: \begin{equation} 25sec^2(5x) = 25 + (5tg(5x))^2. \end{equation}

Utilizando uma propriedade da trigonometria a qual nos mostra que: \begin{equation} 1 + tg^2(t) = sec^2(t). \end{equation}
Então podemos substituir \begin{equation} sec^2(5x) \quad  \textrm{por} \quad 1 + tg^2(5x). \end{equation}
Com isso, temos a equação da seguinte maneira \begin{equation} 25(1 + tg^2(5x)) = 25 + (5tg(5x))^2 \end{equation} aplicando a distributiva no lado esquerdo e abrindo o quadrado do lado direito temos \begin{equation} 25 + 25tg^2(5x) = 25 + 25tg^2(5x), \end{equation} o que nos dá uma identidade, sendo mais visível na simplificação, resultando em 1 = 1
\cite{explicacaoEDO}.

\end{flushleft}
\end{center}

\subsection[Tipos de solução]{Tipos de solução}
Existem três tipos de solução de uma EDO, a \textbf{geral}, a \textbf{particular} e a \textbf{singular}.
\begin{itemize}
	\item{geral:} Onde o número de possíveis constantes é n. Com n da ordem da EDO, a mesma quantidade das unidades da ordem de integração. 
	\item{particular:} É a solução deduzida da solução geral, atribuindo valores particulares a constante, ou seja, o número máximo possível de constantes é 1, com um valor específico.
	\item{singular:} Não é uma solução deduzida da solução geral e só existe em alguns casos.
\end{itemize}

A seguir será mostrado um exemplo de solução geral de uma ED.  \linebreak

			$ \dfrac{dy}{dx} = x \quad  \textrm{é o mesmo que} \quad y' = x, \quad \textrm{com y em função de x => y(x)} \\
			$
			

			integrando dos dois lados, temos que: \linebreak
			
			\begin{equation}
				\int y' dy = \int x dx => y + c1 = \dfrac{x^2}{2} + c2 
			\end{equation}
			
			\begin{center}
				c2 - c1 = C, então temos a solução geral  \\
			\end{center}
			
			\begin{equation}
				y = \frac{x^2}{2} + C
			\end{equation}
		
	
	

Relacionado à solução específica temos os problemas de valor inicial (PVI), onde após encontrar a solução geral, deve-se substituir o valor inicial (VI) na equação para determinar o valor específico da constante.

Exemplo: É fornecido que 
	\begin{equation}y = Ae^{-2x}
	\end{equation}
	é solução para a equação diferencial y' + 2y = 0 e a condição inicial (ou VI) é y(0) = 3 

É possível ver que y realmente é solução da ED.
		Derivando y, temos \begin{equation} y'= -2Ae^{-2x} \end{equation} Substituindo na ED, vemos que: 
		\begin{equation} -2Ae^{-2x} + 2(Ae^{-2x}) = 0  \quad \textrm{ou} \quad 0 = 0\end{equation} sendo assim reduzimos a equação à uma identidade, solucionando a equação.

		Tendo visto que y é solução da ED y'. Vamos agora descobrir a solução particular, substituindo o VI. 

		\begin{equation}
		y(0) = 3 \quad \textrm{e} \quad  y = Ae^{-2x} \newline 
		\end{equation}

		\begin{equation}
		3 = Ae^{-2 * 0}  \quad \textrm{então} \quad
		3 = A
		\end{equation}
		
		Da solução geral \begin{equation} y = Ae^{-2x}  \end{equation}
		\begin{flushleft}temos a solução particular\end{flushleft}  \begin{equation}  y = 3e^{-2x}  \end{equation}
		

\subsection[Equação de variáveis separáveis]{Equação de variáveis separáveis}
São as equações em que um lado da igualdade pode-se separar uma variável e do outro lado a outra variável mais uma constante arbitrária (C). 
Para obter a solução geral de equações separáveis é necessário isolar os termos e integrar os dois lados. Em caso de ser fornecido valor inicial, é possível obter a solução particular.

Exemplo: 
\begin{center}
 Mdx = -Ndy 
\end{center}

Com M = M(x) e N = N(y) podendo assumir funções de uma variável, produto de uma só variável ou constante. 
Abaixo são mostrados exemplos de EDO separáveis:

\begin{table}[!h]
\centering
\begin{tabular}{c | c | c}
\toprule
$
xdx = ydy + C 
$ & 
\begin{tabular}{@{}c@{}c@{}} $ x^2y'y -2xy^3 = 0 $  é igual a \\ $ x^2yy' = 2xy^3 $ e também é igual a \\ $ \dfrac{x^2}{2x} = \dfrac{y^3}{yy'} $\end{tabular} & 
$ xdx + sen(x) = \frac{1}{y}dy - 6y $ \\

\bottomrule
\end{tabular}
\caption{EDO de variáveis separáveis}
\label{edseparaveis}
\end{table}



Abaixo são mostrados exemplos de EDO não separáveis:

\begin{table}[!h]
\centering
\begin{tabular}{c|c|c}
\toprule
\begin{tabular}{@{}c@{}c@{}} $ x^2 -3xy+5y^2 = 0, $ \\ tentando separar, obtemos \\ $ x(x - 3y) = y^2 $\end{tabular}
 & 
 $ (x^2 + y^2 )dx + (x^2 -xy)dy = 0 $
& 
\begin{tabular}{@{}c@{}c@{}} $ x^3 + x^2y +y^3  = 0, $ \\ tentando separar, obtemos \\ $ x^3 + (x^2y) = -y^3 $\end{tabular}  \\

\bottomrule
\end{tabular}
\caption{ED de variáveis não separáveis}
\label{edsnaoeparaveis}
\end{table}

Na tabela \ref{edsnaoeparaveis} vemos que não é tão trivial separar a equações para integrar ambos os lados. Devido às equações de variáveis não separáveis, temos as equações homogêneas para tentar contornar esse problema não separação.

\subsection[Equação Homogênea]{Equação Homogênea}
Algumas EDOs não separáveis podem se tornar separáveis fazendo uma troca de variável. Uma EDO é chamada de homogênea se for satisfeita a seguinte relação:
\begin{equation}
f(kx,ky) = k^m * f(x,y) 
\end{equation}
\begin{center}
com m sendo o grau da homogeneidade.
\end{center}


Equações homogêneas podem ser escritas na forma 

\begin{equation}
Mdx + Ndy = 0  \quad \textrm{,} \quad M(x)\quad  e  \quad N(y)
\end{equation}
\begin{center}
com M e N sendo homogêneas do  mesmo grau.
\end{center}

A seguir um exemplo de EDO não separável e homogênea sendo transformada em uma separável após uma troca de variáveis.
\begin{equation}
f(x,y) = (2x - y )dx - (x + 4y)dy, 
\end{equation}
vamos substituir x por kx e y por ky para verificar que a equação é homogênea

\begin{center}
$ f(kx,ky) = (2kx - ky)dx  - (kx + 4ky)dy$ \\
$ f(kx,ky) = k(2x - y)dx - k(x + 4y)dy  $ \\
$ f(kx,ky) = k[(2x - y) - (x + 4y)] $ 
\end{center}
\begin{equation}
f(kx,ky) = k^1 * f(x,y).
\end{equation}

É uma função homogênea de grau 2

Agora que vimos que é uma função homogênea, podemos fazer uma troca de variável para transformar a função em uma equação de variáveis separáveis. Vamos dizer que y = ux, assumindo que u é uma função diferenciável de x e então $ dy/dx = u + \dfrac{xdu}{dx} $

Então substituindo y e dy na solução e fazendo as manipulações necessárias, temos que: 

\begin{center}
$
	(2x - y )dx - (x + 4y)dy = 0 \linebreak
$
\\
$
	(2x-y)dx = (x+4y)dy  \linebreak
$
\\
$
	\dfrac{(2x-y)}{(x+4y)} = \dfrac{dy}{dx}  \linebreak
$
\\
$
	\dfrac{(2x-(ux))}{(x+4(ux))} = u + \dfrac{xdu}{dx}   \linebreak
$
\\
$
	\dfrac{x(2-u)}{x(1+4u)} = u + \dfrac{xdu}{dx}   \linebreak
$
\\
$
	\dfrac{2-u}{1+4u} - u =  \dfrac{xdu}{dx}   \linebreak
$
\\
$
	\dfrac{\dfrac{2-u}{1+4u} - u}{du} =  \dfrac{x}{dx}   \linebreak
$
\end{center}

Sendo assim, vimos um exemplo de como transformar uma EDO de variáveis não separáveis em uma EDO de variáveis separáveis, agora basta integrar os lados da igualdade.


\subsection[Equação Exata]{Equação Exata}

Uma EDO é denominada exata se puder satisfazer duas condições: 

\begin{itemize}
\item{1ª:} ser escrita na forma de: 
\begin{equation}
Mdx + Ndy = 0;
\end{equation}
Com M = M(x,y) e N = N(x,y).
\item{2ª:} estabelecer a seguinte igualdade:
\end{itemize}

\begin{center}
$ \dfrac{ \partial M}{ \partial y} =  \dfrac{ \partial  N}{ \partial x} $
\end{center}

Algumas vezes, uma função não exata pode ser transformada em exata, multiplicando-a por um fator de integração $\mu(x)$, que resulta em : 

\begin{center}
$ \mu M(x,y)dx + \mu N(x,y)dy = 0 $
\end{center}

\subsection[Equação Linear 1ª ordem]{Equação Linear 1ª ordem}
Uma equação linear de 1ª ordem pode ser definida da forma geral como: 

\begin{equation} \label{formaLinear}
  \operatorname{a}_{1}(x)\dfrac{dy}{dx} + \operatorname{a}_{0}(x)y = g(x) 
\end{equation}

dividindo toda a equação por a1, teremos   $ \operatorname{a}_{0}(x)/\operatorname{a}_{1}(x) $  uma função P(x)  e $ g(x)/\operatorname{a}_{1}(x) $ uma função f(x). 
Reescrevendo a equação, teremos dy/dx + P(x)y = f(x). Multipliquemos agora toda a equação por dx e passemos o termo f(x) para o lado esquerdo da equação e teremos
\begin{center}
dy + (P(x)y - f(x))dx = 0
\end{center}

Multiplique a equação por $\mu(x)$

\begin{center}
$\mu(x)$dy + $\mu(x)$(P(x)y - f(x))dx
\end{center}

Pelo critério para ser uma ED exata citado em \cite{explicacaoEDO}, a equação é uma diferencial exata se 

\begin{center}
$ \dfrac{\partial}{\partial x}\mu(x) = \dfrac{\partial}{\partial y}\mu(x)(P(x)y - f(x))  $ 
\end{center}


Do lado esquerdo temos uma derivada ordinária e do lado direito derivamos em y. 

\begin{center}
$ \dfrac{d\mu(x)}{dx} = \mu(x)P(x) $
\end{center}

Agora vamos multiplicar a equação por $ \dfrac{dx}{\mu} $ para obter uma ED separável

\begin{center}
$ \dfrac{d\mu}{\mu} = P(x)dx $
\end{center}


Para resolver a equação separável, integramos ambos os lados

\begin{center}
$ \bigints \dfrac{d\mu}{\mu} = \bigints P(x)dx $
\end{center}

Como resultado, obtemos

\begin{center}
$ ln|u| =  \bigints P(x)dx $
\end{center}

Multiplicando por e de ambos os lados

\begin{center}
$ e^{ln|\mu|} = e^{\bigints P(x)dx} $
\end{center}

Dessa maneira encontramos o fator integrante como sendo

\begin{center}
$ \mu(x) = e^{\bigints P(x)dx} $
\end{center}

Dessa maneira podemos resolver equações diferenciais exatas e equações diferenciais de primeira ordem
