\begin{resumo}
O objetivo deste trabalho foi desenvolver um jogo de celular o AprEnDO, para analisar o
 apoio à aprendizagem de equações diferenciais ordinárias (EDO) de 1ª ordem.
 O jogo contém perguntas a respeito de classificação e resolução dessas equações. A metodologia
 de trabalho foi um relato de experiência de uma aplicação de jogo em uma classe de Cálculo 2 (C2) da Faculdade do Gama da UnB em que a professora orientadora ministrou o ensino. Em 1/2019 a turma começou a utilizar o jogo e foram aproximadamente 2 meses de coletas em um servidor de dados para relatar a experiência do AprEnDO em sala de aula. Ao final do trabalho, são relatados os resultados dos dados analisados e as pendências para trabalhos futuros.
	
\begin{comment}
 Com a qualidade de ensino de matemática baixa e contra os métodos muito
 tradicionais de ensino nas salas de aula, resolveu-se desenvolver um 
 aplicativo para celular (iOS e Android) que seja um jogo para o suporte de 
 ensino de equações diferenciais (ED). Será realizada uma pesquisa descritiva
 para o levantamento bibliográfico das características que deverão estar presentes
 no software para dar auxílio a alunos com TDAH junto de técnicas de gamificação 
 para tentar deixar o aprendizado mais prazeroso.
\end{comment}
  
 \begin{comment}
 O resumo deve ressaltar o objetivo, o método, os resultados e as conclusões 
 do documento. A ordem e a extensão
 destes itens dependem do tipo de resumo (informativo ou indicativo) e do
 tratamento que cada item recebe no documento original. O resumo deve ser
 precedido da referência do documento, com exceção do resumo inserido no
 próprio documento. (\ldots) As palavras-chave devem figurar logo abaixo do
 resumo, antecedidas da expressão Palavras-chave:, separadas entre si por
 ponto e finalizadas também por ponto. O texto pode conter no mínimo 150 e 
 no máximo 500 palavras, é aconselhável que sejam utilizadas 200 palavras. 
 E não se separa o texto do resumo em parágrafos.
 \end{comment}

 \vspace{\onelineskip}
    
 \noindent
 \textbf{Palavras-chave}: Equação Diferencial. Aplicativo para celular. Jogo. Software.
\end{resumo}
