\begin{resumo}

	O objetivo deste trabalho é desenvolver um jogo de celular para apoiar a aprendizagem de equações diferenciais ordinárias (ODE). A metodologia de trabalho vai ser um estudo de caso aplicado em classes de Cálculo 2 (C2) e será comparado com outras classes que não tiveram contato com o jogo. As turmas envolvidas são C2 da Faculdade do Gama da UnB tendo a professora orientadora como uma das professoras participantes do estudo. Os feedbacks serão analisados a partir dos dados do jogo enviados pelos jogadores além da aplicação dos questionários para analisar se o jogo contribuiu para o aprendizado efetivo.
	
\begin{comment}
 Com a qualidade de ensino de matemática baixa e contra os métodos muito
 tradicionais de ensino nas salas de aula, resolveu-se desenvolver um 
 aplicativo para celular (iOS e Android) que seja um jogo para o suporte de 
 ensino de equações diferenciais (ED). Será realizada uma pesquisa descritiva
 para o levantamento bibliográfico das características que deverão estar presentes
 no software para dar auxílio a alunos com TDAH junto de técnicas de gamificação 
 para tentar deixar o aprendizado mais prazeroso.
\end{comment}
  
 \begin{comment}
 O resumo deve ressaltar o objetivo, o método, os resultados e as conclusões 
 do documento. A ordem e a extensão
 destes itens dependem do tipo de resumo (informativo ou indicativo) e do
 tratamento que cada item recebe no documento original. O resumo deve ser
 precedido da referência do documento, com exceção do resumo inserido no
 próprio documento. (\ldots) As palavras-chave devem figurar logo abaixo do
 resumo, antecedidas da expressão Palavras-chave:, separadas entre si por
 ponto e finalizadas também por ponto. O texto pode conter no mínimo 150 e 
 no máximo 500 palavras, é aconselhável que sejam utilizadas 200 palavras. 
 E não se separa o texto do resumo em parágrafos.
 \end{comment}

 \vspace{\onelineskip}
    
 \noindent
 \textbf{Palavras-chave}: ED. aplicativo para celular. jogo. software.
\end{resumo}
