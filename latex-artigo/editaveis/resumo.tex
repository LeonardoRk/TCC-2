\begin{resumo}
O objetivo deste trabalho era desenvolver um jogo de celular o aprEnDO para analisar se
 foi possível apoiar a aprendizagem de equações diferenciais ordinárias (EDO) de 1ª ordem.
 O jogo contém perguntas a respeito de classificação e resolução de equações. A metodologia
 de trabalho foi um estudo de caso aplicado em  uma classe de Cálculo 2 (C2) da Faculdade 
 do Gama da UnB em que a professora orientadora ministra o ensino. Um grupo aleatório 
 utilizará o aplicativo enquanto outro grupo não terá contato com o jogo. Os feedbacks 
 serão analisados a partir dos dados do jogo enviados pelos jogadores além da aplicação e 
 análise de mapas conceituais para analisar se o jogo contribuiu para o aprendizado efetivo 
 da matéria.
	
\begin{comment}
 Com a qualidade de ensino de matemática baixa e contra os métodos muito
 tradicionais de ensino nas salas de aula, resolveu-se desenvolver um 
 aplicativo para celular (iOS e Android) que seja um jogo para o suporte de 
 ensino de equações diferenciais (ED). Será realizada uma pesquisa descritiva
 para o levantamento bibliográfico das características que deverão estar presentes
 no software para dar auxílio a alunos com TDAH junto de técnicas de gamificação 
 para tentar deixar o aprendizado mais prazeroso.
\end{comment}
  
 \begin{comment}
 O resumo deve ressaltar o objetivo, o método, os resultados e as conclusões 
 do documento. A ordem e a extensão
 destes itens dependem do tipo de resumo (informativo ou indicativo) e do
 tratamento que cada item recebe no documento original. O resumo deve ser
 precedido da referência do documento, com exceção do resumo inserido no
 próprio documento. (\ldots) As palavras-chave devem figurar logo abaixo do
 resumo, antecedidas da expressão Palavras-chave:, separadas entre si por
 ponto e finalizadas também por ponto. O texto pode conter no mínimo 150 e 
 no máximo 500 palavras, é aconselhável que sejam utilizadas 200 palavras. 
 E não se separa o texto do resumo em parágrafos.
 \end{comment}

 \vspace{\onelineskip}
    
 \noindent
 \textbf{Palavras-chave}: ED. aplicativo para celular. jogo. software.
\end{resumo}
