\begin{resumo}[Abstract]
 \begin{otherlanguage*}{english}     
   The objective is develop a cellphone game to support ordinary differential equation (ODE) learning. The methodology of work is gonna be a case study applied in classes of Calculum 2 (C2) and will be compared with others classes that did not had in touch with the game. Involved classes are C2 of the Faculty of Gama of UnB having the mentor teacher as one of the participant teacher of the study. Feedbacks will be analyzed from the game's data sent by the players besides the questionaries' application to analyze if the game contributed to the effective learning.
   
   \begin{comment}
   metodologia 
   revisão bibliográfica 
   estudo de caso 
      software educacional 
   ensino equações diferenciais ordinárias
   auxílio de aprendizagem de EDO
   Faculdade do Gama
   diagnostico através da avaliação
   aplicação de questionário
   \end{comment}
   

   \vspace{\onelineskip}
 
   \noindent 
   \textbf{Key-words}: differential equation; mobile game; ADHD.
 \end{otherlanguage*}
\end{resumo}
