\begin{resumo}[Abstract]
 \begin{otherlanguage*}{english}     
The purpose of this paper was to develop the AprEnDO, a cellphone game, to analyze the learning from first order ordinary differential equation (ODE). The game presents questions about equation's classification and resolution. The methodology was an experience report of the game application in Cálculus (C2) classes of Gama's College in the UnB which the mentor teacher ministered the teaching. On the date 2019/1 first semester the class started using the game and for approximately two months a server stayed collecting games's data to report the AprEnDO's experience in classroom and were verified low students participation. As valitation method to knowledge organization in the cognitive structure there was the use of concept map. 

      
   \vspace{\onelineskip}
 
   \noindent 
   \textbf{Key-words}: Differential Equation. Mobile app. Game. Software.
 \end{otherlanguage*}
\end{resumo}
