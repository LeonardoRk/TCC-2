\chapter[Introdução]{Introdução}

\begin{comment}
Tendo em vista que já é difícil para alunos aprenderem matemática, a dificuldade só aumenta quando fala-se de alunos com Transtorno do Déficit de Atenção com Hiperatividade (TDAH) inseridos no contexto de faculdade, onde a desatenção aumenta, já que professores não estão preparados e precisam renovar seus métodos para atrair atenção dos alunos para ensinar a matéria e ainda fazer disso algo divertido para fazer com que o foco seja maior ainda. Já que aprender brincando gera melhores resultados. Segundo (Russel A. Barkley, PhD. p. iv) quem possui TDAH têm mais dificuldades que pessoas normais em ambientes que exijam mais foco, objetividade e autocontrole. Também é dito que as características principais de TDAH podem trazer diversas dificuldades no contexto escolar (George J. DuPaul, PhD e Gary Stoner, PhD. p.4).

De acordo com (George J. DuPaul, PhD e Gary Stoner, PhD. p.4) o TDAH comparado a outros problemas como autismo e depressão é um transtorno de alta incidência e se mostra presente principalmente em meninos. Por isso o foco deste estudo também será em estudantes com TDAH.
\end{comment}

Como será mostrado a seguir no estudo, descobrimos que o Brasil tem uma qualidade de ensino de matemática inferior a de muitos países, inclusive, foi classificado abaixo do nível do que é considerado o nível mínimo para exercer a cidadania como cidadão pleno \cite{inep2015nivelcidadania}. Avaliando que a média das menções de Cálculo 2 (C2) na UnB está muito concentrada no Médio (MM), levando em consideração o estudo \cite{evasaoC2} que diz que C2 é uma disciplina das que mais causa a evasão dos alunos do curso de matemática noturno na UnB e que existem poucos trabalhos a respeito de como ensinar equações diferenciais (ED) para estudantes, decidiu-se fazer um jogo para celular com o intuito de inserir no ambiente uma ferramenta a mais para ajudar os estudantes a aprender divertindo, já que tem-se estudos reforçando que o jogos ajudam e aumentam as chances de aprendizado, contribuem para incluir estudantes com deficiências no meio em que estão inseridos, tornando-o mais pró ativos e melhorando suas capacidades de se articularem. Existem também estudos mostrando como a tecnologia da suporte para jogos na hora do ensino e ajuda na fixação do conhecimento. A metodologia seguida foi a pesquisa bibliográfica para levantar o referencial teórico e ajudar com técnicas para o desenvolvimento do jogo que visa treinar os alunos a reconhecer, classificar, resolver e aplicar equações diferenciais presentes no dia-a-dia da engenharia.


O foco deste trabalho é desenvolver um jogo para celular Android e iOS com estratégias gamificadas para dar suporte à fixação do conhecimento em ED.
O jogo conterá 3 fases com níveis de dificuldades que treinem a classificação de ED, resolução de exercícios e aplicações no dia a dia e no ambiente da engenharia. Ao fim de todos os níveis de dificuldades de cada fase tem-se o desafio para completar a categoria.
O jogo será planejado e desenvolvido utilizando metodologias ágeis de software. As funcionalidades (features) e a descrição granulariazada (histórias de usuários - HU) serão elencadas e descritas para que se tenha a rastreabilidade dos requisitos do jogo. Com o jogo pronto será aplicado em uma turma de C2 no período do primeiro semestre de 2019 para que possa ser gerado um diagnóstico avaliativo concluindo se o jogo trouxe alguma eficiência no aprendizado ou não.
O capítulo 2 abordará o referencial teórico, dando ênfase nos baixos indíces de classificação do Brasil no conhecimento de matemática e apoiando a gamificação e jogos como uma prática que deixa as tarefas e atividades mais divertidas.
O capítulo 3 explica o problema existente e a justificativa do trabalho. O capítulo 4 aborda a respeito da questão de pesquisa e os objetivos gerais e específicos. O capítulo 5 aborda ED para introduzir um nivelamento de conteúdo a ser abordado no jogo. O capítulo 6 explica a metodologia do trabalho. capítulo 7 explica as fases do jogo, como espera-se que ele seja jogado. capítulo 8. fala a respeito das tecnologias utilizadas para desenvolvimento e do planejamento do jogo. Capítulo 9 apresenta a conclusão do trabalho e o capítulo 10 mostra as referências do trabalho e em seguida os apêndices e anexos.


\begin{comment}
O aplicativo deverá ser um jogo com estratégias gamificadas para o ensino de aplicações e práticas do conteúdo de equação diferencial ordinária (EDO) de primeira ordem aos alunos do ensino superior com TDAH. 
\end{comment}
