\chapter[Introdução]{Introdução}


Sabe-se que os jogos propiciam diversão ao seres humanos e vê-se que a gamificação está sendo explorada em várias áreas diversas como marketing, saúde e educação. Segundo \cite{revbibmatgam} as tecnologias digitais em sala de aula podem ser instrumentos que auxiliam no processo de ensino e de aprendizagem. Quem sabe então se os jogos e os aplicativos gamificados não podem ser usados na educação para ajudar mais pessoas a aprender e se interessarem pelo assunto de Matemática ou qualquer outro. Será que usar jogos digitais para o ensino de Matemática pode ajudar os alunos a se interessarem e aprenderem mais? Se sim, como utilizar? 

Existem relatos de trabalhos que usam jogos e gamificação na matemática. Mas a maioria do conteúdo de Matemática para o ensino superior encontrado era cálculo 1.  No Brasil os jovens tem dificuldades com matemática. Segundo o relatório do \cite{inep2015nivelcidadania} o Brasil tem uma qualidade de ensino de matemática inferior a de muitos países. A Organização para a Cooperação e Desenvolvimento Econômico (OCDE) utiliza uma escala de classificação que vai de 1 a 6 para as habilidades de matemática e de acordo com o conjunto da população brasileira que participou da pesquisa no Programa Internacional de Avaliação de Estudantes (PISA), inferiu-se que 70.3\% dos estudantes brasileiros estão abaixo do nível 2, o qual foi estabelecido como o mínimo para exercer a cidadania como cidadão pleno \cite{inep2015nivelcidadania}.
O estudo é realizado a cada três anos. Foi realizado no ano passado, porém até o momento o resultado ainda não foi publicado \url{http://portal.inep.gov.br/web/guest/acoes-internacionais/pisa/resultados}.

Além das informações relatadas acima outras três também contribuíram para formular a proposta deste trabalho. Um deles foi o estudo de \cite{revbibmatgam}, que é uma revisão sistemática de literatura realizada nas bases de dados Scielo Library, Science Direct, ACM Library e IEEE Xplore Digital Library entre outras e nos periódicos RBIE e a RENOTE. O estudo procurou relatos em artigos da existência de ferramentas relacionadas com gamificação que abordem dificuldades de aprendizagem de Matemática e/ou Discalculia. De 2008 trabalhos selecionados, nenhum eram relacionando gamificação e dificuldades de matemática ao mesmo tempo. Com essas conclusões encontradas no estudo os autores \cite{revbibmatgam} concluíram que:

\begin{citacao}"identifica-se a necessidade de pesquisas sobre esta temática (gamificação com dificuldades de aprendizagem de Matemática), já que as dificuldades de aprendizagem na Matemática são frequentes em sala de aula, e a gamificação tem-se mostrado uma ferramenta promissora nos ambientes de ensino e aprendizagem em todos os níveis de ensino" \cite{revbibmatgam}. \end{citacao}

No mesmo estudo de \cite{revbibmatgam} o autor cita \cite{dicheva} dizendo que "a falta de pesquisa na área é justificada por ser uma temática nova.", o que pode aumentar com o tempo caso as pessoas se interessem por essa 'temática nova' de utilizar gamificação na educação e para o ensino de matemática.


Outro fator foi o estudo de \cite{evasaoC2} o qual diz que Cálculo 2 (C2) é uma disciplina das que mais causa a evasão dos alunos do curso de Matemática noturno na UnB. Após saber da existência do estudo no site de monografias da unb houve a comparação da ementa de C2 no curso de Matemática noturno do Darcy Ribeiro com a ementa de C2 do tronco comum no curso de Engenharias da UnB no Gama e foi constatado que há a equivalência dos conteúdos ministrados das disciplinas. Ambas iniciam para os alunos o conteúdo de equações diferenciais. Assunto que por acaso é pouco encontrado na literatura de gamificação. O mais encontrado é voltado para área de limite, derivada e integral como mostrado o aplicativo em \cite{appcalculo} e também presente na própria FGA. Verificou-se que há poucos jogos de Matemática para o conteúdo de equação diferencial. O único jogo de equações diferenciais encontrado é um de vídeo-game que aplica as equações diferenciais na movimentação dos personagens (princípio da dinâmica de Newton)\cite{videoGameED}. A maioria dos jogos de Matemática é para o ensino fundamental.


Existem também estudos mostrando como a tecnologia da suporte para jogos na hora do ensino e ajuda na fixação do conhecimento. 

Tendo relatado as dificuldades no ensino de matemática e a possível ajuda de gamificação e jogos digitais, gerou-se a questão: Como dar suporte no ensino de EDO 1ª ordem apresentados em sala de aula de forma lúdica? A partir deste questionamento, levantou-se o objetivo: desenvolver um jogo para celular Android que dê suporte ao ensino de equações diferenciais ordinárias (EDO) de 1ª ordem. O jogo visa treinar os alunos a reconhecer, classificar e resolver equações diferenciais presentes no dia-a-dia e no ambiente da engenharia.

Decidiu-se fazer um jogo para celular com o intuito de inserir no ambiente dos alunos uma ferramenta a mais (o jogo) para ajudar os estudantes a aprenderem mais e se possível se divertindo, ou que pelo menos seja interessante. Deseja-se que seja um meio de treinamento e fixação do conhecimento para aprenderem.

A primeira parte da metodologia seguida foi a pesquisa bibliográfica para levantar o referencial teórico a respeito da contribuição efetiva de jogos e seu sucesso no ensino. Já a segunda parte da metodologia foi um relato de experiência do desenvolvimento do software e a aplicação em alunos de C2 na universidade do Gama, utilizando estatísticas levantadas do jogo chamado aprEnDO para avaliar o impacto do aplicativo educacional para celular como ferramenta de suporte didatico no ensino principalmente em classificação de EDO de 1º nível.

Utilizou-se o processo de prototipação de baixa fidelidade para desenhar as telas do jogo pois ela serve como auxílio para uma das etapas do processo de desenvolvimento de software que é a elicitação dos requisitos. Houve a consolidação dos requisitos funcionais que foram definidos e a modelagem de um diagrama de classe para explicar a aplicação AprEnDO que é um projeto do \textit{react native}. São indicados os domínios utilizados para baixar as dependências que compõem o software e toda a parte da hospedagem dos códigos tanto da aplicação, como para gerar o arquivo de alimentação e os que definiram o servidor que guarda os dados das estatísticas enviadas pelos alunos. O projeto pode ser reutilizado por quem desejar e ajudas na contribuição são bem vindas. Com o jogo pronto será aplicado em um grupo da turma de C2 no período do primeiro semestre de 2019 para que possa ser gerado dados e estatísticas para concluir se o jogo trouxe alguma eficiência no aprendizado ou não.

O jogo contém três módulos. Dois deles para jogar com diferentes fases e dificuldades e o outro restante serve para envio de estatísticas dos dados dos jogadores para um servidor com um banco de dados. Foi desejado planejá-lo para ser 'fácil' de enviar os dados colhidos para análise. É chamada de fácil por ser considerada rápida do jogador utilizar. Exige-se apenas a matrícula do aluno (que será validada) e a confirmação de um clique para acordar compartilhar suas estatísticas de variações entre o tempo atual e o último envio.

Houve uma falta de planejamento no desenvolvimento de software devido aos atrasos de elicitação dos requisitos o que impossibilitou a completude de alguns e como o jogo havia data certa para início de aplicação impossibilitou o término do desenvolvimento de alguns requisitos. 

O capítulo 2 abordará o referencial teórico, dando ênfase nos baixos indíces de classificação do Brasil no conhecimento de matemática, apoiando a gamificação e jogos como uma prática que deixa as tarefas e atividades mais divertidas, revisando conceitos de engenharia de software para desenvolvimento de jogos.
O capítulo 3 explica a metodologia do trabalho seguida. O capítulo 4 fala a respeito da engenharia de software aplicada no projeto, cita requisitos do jogo, artefatos de auxílio, ambiente de desenvolvimento, o banco de dados, como é o processo de empacotamento e o servidor que recebe dados de jogo. O capítulo 5 é o manual do AprEnDO que explica as fases do jogo, como espera-se que ele seja jogado e fotos do aplicativo. O Capítulo 6 mostra a análise dos dados colhidos. O capítulo 7 apresenta a conclusão do trabalho e o possíveis atividades futuras e o capítulo 7 mostra as referências do trabalho.
